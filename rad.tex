% TODO: Navedite tip rada (diplomski, zavrsni, seminar). Vidi docs folder za upute ili templates/template.cls.
\documentclass[times, utf8, diplomski, numeric]{templates/template}
\usepackage{booktabs}

\begin{document}

% TODO: Dodaj imena sveucilista i fakulteta
\sveuciliste{SVEUČILIŠTE U ZAGORJU}
\fakultet{FAKULTET ELEKTROTEHNIKE I DOSADE}

% TODO: Dodaj ime i prezime voditelja. 
% Samo ako pisete seminar (inace ignorirajte ili izbrisite)
\voditelj{Ime i prezime voditelja}

% TODO: Dodaj naslov rada
\title{Ovo je naslov rada}

% TODO: Dodaj broj rada (ignoriraj ako pisete seminar)
\thesisnumber{001}

% TODO: Dodaj ime i prezime (ili vise njih odvojene zarezima)
\author{Ime i prezime autora}

\maketitle

% Ispis stranice s napomenom o umetanju izvornika rada. 
% Ignorirajte ako pisete seminar.
% Uklonite naredbu \izvornik ako želite izbaciti tu stranicu.
\izvornik

% TODO: Dodaj zahvalu
% Ignorirajte ako pisete seminar.
% Uklonite naredbu \zahvala ako zelite izbaciti tu stranicu.
\zahvala{Ovdje ide zahvala}

\tableofcontents

% TODO: Dodaj Uvod
\chapter{Uvod}
Ovdje ide uvod.

% Citate dodaj u `literatura.bib' file
\chapter{Kako citirati}
Ovako se citira na pravilan nacin \cite{oetiket2007lshort}. I onda jos malo teksta \cite{downes2002shortams}. 
Evo jos malo teksta kojeg citiramo ovako \cite{ungar2002uvod}.

% TODO: dodaj zakljucak
\chapter{Zaključak}
Ovdje ide zaključak.

\bibliographystyle{templates/template}
\bibliography{literatura}

% TODO: dodaj sazetak
\begin{sazetak}
Ovdje ide sažetak na hrvatskom jeziku.

% TODO: dodaj kljucne rijeci
\kljucnerijeci{Ovdje, idu, ključne, riječi, odvojene, zarezima.}
\end{sazetak}

% TODO: Dodaj naslov na engleskom
\engtitle{Put here title on english}
% TODO: Navedite Abstract
\begin{abstract}
Add abstract here.

% TODO: Navedite Keywords
\keywords{Add, keywords, here.}
\end{abstract}

\end{document}
