\documentclass[times, utf8, diplomski, numeric]{templates/template}
\usepackage{booktabs}

\begin{document}

\sveuciliste{SVEUČILIŠTE U ZAGREBU}
\fakultet{FAKULTET ELEKTROTEHNIKE I RAČUNARSTVA}

\title{Programska podrška sustava za određivanje i upravljanje orijentacijom satelita}

\thesisnumber{2346}

\author{Ivan Vnučec}

\maketitle

% Ispis stranice s napomenom o umetanju izvornika rada. 
\izvornik

% TODO: Dodaj zahvalu
\zahvala{Ovdje ide zahvala}

\tableofcontents

\nocite{*}
\chapter{Nanosateliti}{
    \section{Uvod}{
        - sto su Nanosateliti
        - zasto postoje, 
            - za sto se koriste
            - tko ih koristi
    }

    \section{Podjela prema veličini}{
        \subsection{Cubesat format}{
            - sto je cubesat satelit
            - od cega se sastoji, 
                - gradja, 
                - nacin slaganja dodatnih modula
            - kojih sve verzije cubesat satelita postoje
            - zasto bas kocka
            - prednosti i mane cubesat satelita
            - navesti negdje brojke o udijelu cubesat sateltia u ostalim satelitima kroz godine
            - zasto je popularan cubesat format
        }
    }

    \section{Metode lansiranja}{
        - objasniti kako se sateliti lansiraju
            - objasni da oni nisu glavni payload
            - objasni kako su pakirani
            - objasni u koje orbite mogu ici
            - objasni tvrtke koje se bave lansiranjem
        - navedi koliko najcesce kosta lansiranje
        - navedi kako najlakse iznajmit lansiranje
        - rast lansiranja nanosateltia iz godine u godinu
            - pronaci negdje neke brojke o lansiranju cubesat satelita po godinama
    }

    \section{Povijesni pregled misija}{
        - navesti neke komercijalne primjene do sada
        - navesti znanstvene misije do sada
    }
}


\chapter{Osnovni kinematički model satelita}{
    \section{Matematički zapis}{
        - objasniti sve parametre kinematickog modela satelita
        - objasniti pojednostavljenja
        - objasniti odredjivanje parametara
    }
    
    \section{Orijentacija satelita}{   
        \subsection{Vrste reprezentacija}{
            \subsubsection{eulerovi kutevi (ili kutovi)}{

            }

            \subsubsection{kvaternioni}{

            }
        }

        \subsection{Usporedba vrsta reprezentacija}{

        }
    }
    
    \section{Kutna brzina satelita}{
        - sto je kutna brzina satelita
        - kako ju reprezentiramo
            - promjena eulerovih kuteva
            - promjena kvaterniona
    }
}
    
\chapter{Sustav za odredjivanje i upravljanje orijentacijom satelita}{
    \section{Uvod}{
        - sto je sustav za odredjivanje i upravljanje orijentacijom satelita
            - navesti englesku skracenicu (koristi eng u latexu, vidi dokumentaciju pri dnu)
        - koja je zadaca sustava za odredjivanje a koja sustava za upravljanje orijentacijom satelita
        - zasto je on jedan od najvaznijih sustava satelita
            - sto se desava ako sustav prestane raditi ili radi nenormalno
                - navesti neke primjere
        - Zasto nam je bitna orijentacija
            - korisni teret satelita
                - navesti neke korisne terete/sustave koji ovise o orijentaciji satelita
                - u ovisnosti o teretu, objasniti koje vrste orijentacije koristimo (npr antena prati zemlju, itd)
    }

    \section{Osnovni dijelovi}{
        \subsection{Računalo}{
            - prikupljanje podataka
            - obrada podataka
            - kontrola orijentacije
        }

        \subsection{Senzori}{
            - koji sve postoje
            - na kojem principima rade
            - navesti neke tehnicke podatke svakog
            - napraviti usporedbu izmedju njih
                - cijena
                - preciznost
                - uvijeti rada
                - dostupnost
                - jednostavnost
                - potrosnja
        }

        \subsection{Aktuatori}{
            - koji sve postoje
            - na kojim principima rade
            - tehnicki podatci
            - usporedba izmedju njih
                - cijena
                - preciznost
                - uvijeti rada
                - dostupnost
                - jednostavnost
                - potrosnja
        }
    }

    \section{Određivanje orijentacije}{
        - kako mozemo estimirati orijentaciju satelita
            - senzor fusion
        - kako mjerimo kutnu brzinu
            - ziroskop
            - sto je ziroskop
            - problemi ziroskopa
    }

    \section{Upravljanje orijentacijom}{
        - ukratko o upravljanju orijentacijom
        - ukratko o upravljanju kutnom brzinom

        - navesti aktuatore
            - zamasnjake, magnetorqere
            - navesti problem zasicenja zamasnjaka

        - regulacija
            - navesti PID regulator kao regulator
            - opisati sto je PID regulator
            - navesti quaternion regulator kao regulator
                - jednadzbe
                - dodaj referencu na paper
    }
}

\chapter{Razvijeni ADCS sustav}{
    \section{Senzori}{
        - zasto koristimo IMU
            - napisati da akcelerometar ne moze raditi u svemiru
            - mane ziroskopa
                - bias
            - mane akcelerometra
                - sum
            - magnetometar
                - model magnetskog polja zemlje
                    - mijenja se s vremenom
    }

    \section{Aktuatori}{
        - zasto koristimo bas (stavi ime) aktuator 
            - a ne magnetorqer npr
            - za magnetorqer navesti referencu na izracun parametara koji je radio Ivan Indir
            - navesti karlin rad, pitaj ju sto je ona radila pa ju citiraj
    }

    \section{Matematicki zapis orijentacije}{
        - jednadzbe koje koristimo u SW
    }

    \section{Metoda estimacije orijentacije}{
        - objasniti na koji nacin radimo senzor fuzion
            - optimal request
                - napisati nesto kratko o OR algoritmu
                - navesti referencu na svu popratnu dokumentaciju
    }

    \section{Metoda upravljanja orijentacijom}{
        - odabrana metoda upravljanja orijentacijom
            - spomeni quaternion regulator
        - odabrana metoda upravljanja kutnom brzinom
            - spomeni zamasnjak
            - pwm
            - spomeni PID regulator
            - PID regulator za kontrolu orijentacije satelita
                - opisati kako ga mi koristimo
                - ovo isto pokupi od Arduino projekta
    }

    \section{Sklopovlje}{
        - ukratko o razvijenoj plocici
        - staviti kakvu shemu
    }

    \section{Programska podrška}{
        \subsection{Opis ugradbenog racunala}{
            - koje ugradbeno racunalo koristimo
            - zasto smo bas to odabrali
        }

        \subsection{Opis programskog koda}{
            - objasniti koji compiler koristimo
            - objasniti koji programski jezik koristimo
            - objasniti da je cijeli kod na githubu
            - broj linija koda
                - vidi kako to dobiti u linux-u sa nekom komandom
        }

        \subsection{Organizacija}{
            - objasniti organizaciju koda
        }

        \subsection{Funkcionalnost}{
            - koje biblioteke koristimo i zasto
            - opisati da koristimo FreeRTOS
            - opisati koje sve dretve imamo
                - za svaku dretvu napisati dijagram toka
                - opisati funkciju dretve
                - opisati vrijeme izvodjenja dretvi
                - opisati na koji nacin dretve medjusobno komuniciraju
        }

        \subsection{Razvoj}{
            - objasniti koje sve alate koristimo prilikom razvoja SW
                - clang format i slicno
                - vscode
                - openocd
            - objasniti kako debugiramo
            - slikati graf razvoja programskog koda kroz vrijeme
        }
    }
}

\chapter{Eksperimentalna verifikacija ADCS sustava}{
    \section{Opis sustava}{
        - opisati zracni lezaj
        - opisati kuglu
        - opisati kako satelit stoji u kugli
        - navesti reference za zracne lezaje
        - parser
        - program za iscrtavanje
        - bluetooth modul za komunikaciju
        - iscrtavanje orijentacije papirnatog avioncica u Octave programu
    }

    \section{Odredjivanje parametara kinematickog modela satelita}{
        - objasni kako smo dobili parametre
    }

    \section{Optimizacija PID regulatora}{
        - vidi cubesat na arduinu
        - objasni kako smo dobili parametre PID regulatora
    }

    \section{Qkako racunamo razlike trenutne i zeljene orijentacije}{
        - razlika eulerovi kutova
            - staviti formulu iz matlaba
        - razlka kvaterniona
    }

    \section{Rezultati verifikacije}{
        - ovo nezz
    }
}

% TODO: dodaj zakljucak
\chapter{Zaključak}{
    Ovdje ide zaključak.
}

\bibliographystyle{templates/template}
\bibliography{literatura}

% TODO: dodaj sazetak
\begin{sazetak}{
    Ovdje ide sažetak na hrvatskom jeziku.
}

% TODO: dodaj kljucne rijeci
\kljucnerijeci{Ovdje, idu, ključne, riječi, odvojene, zarezima.}
\end{sazetak}

% TODO: Dodaj naslov na engleskom
\engtitle{Put here title on english}
% TODO: Navedite Abstract
\begin{abstract}{
    Add abstract here.
}

% TODO: Navedite Keywords
\keywords{Add, keywords, here.}
\end{abstract}

\end{document}
