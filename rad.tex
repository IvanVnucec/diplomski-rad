\documentclass[times, utf8, diplomski, numeric]{templates/template}
\usepackage{booktabs}

\begin{document}

\sveuciliste{SVEUČILIŠTE U ZAGREBU}
\fakultet{FAKULTET ELEKTROTEHNIKE I RAČUNARSTVA}

\title{Programska podrška sustava za određivanje i upravljanje orijentacijom satelita}

\thesisnumber{2346}

\author{Ivan Vnučec}

\maketitle

% Ispis stranice s napomenom o umetanju izvornika rada. 
\izvornik

% TODO: Dodaj zahvalu
\zahvala{Ovdje ide zahvala}

\tableofcontents

% TODO: delete me
\nocite{*}

\chapter{Uvod}{
    \section{Mali sateliti}{
        %- sto su mali sateliti
        Mali sateliti su skupina satelita koji su, u usporedbi sa konvencionalnim satelitima, uvelike ograničeni prema masi, cijeni, kao i prema vremenu potrebnom da ih se razvije. Sve je to izravan rezultat revolucije u elektronici i računarstvu.\cite{hrvatskiVojnik}.

        %- tko ih koristi
        Primjene malih satelita nalazimo u svim granama svemirske industrije: u telekomunikaciji, navigaciji, opservaciji Zemlje, znanstvenim istraživanjima i dr.
        
        %- zasto ih koristimo
        Njihova zastupljenost na tržištu raste iz godine u godinu jer pružaju jeftin i brz razvoj uz relativno male tehničke zahtjeve \cite{rastMalihSatelita}. Primjenom većeg broja malih satelita povezanih u konstelaciju, moguće je primjerice pokriti veću površinu zemlje i tako stvoriti uvijek dostupni internet ili navigacijski sustav.
        
        \subsection{Podjela}{
            Iako postoje više vrsta podjela satelita u ovisnosti od izvora do izvora, navest ćemo podjelu prema FAA \engl{Federal Aviation Administration} koja uključuje satelite sa najvećom, pa sve do satelita sa najmanjom masom \cite{podjelaPremaMasi}. Podjelu je moguće vidjeti u tablici \ref{tbl:podjelaSatelita}.

            \begin{table}[htb]
            \caption{Podjela malih satelita prema masi}
            \label{tbl:podjelaSatelita}
            \centering
            \begin{tabular}{ll} \toprule
            Naziv kategorije satelita & Masa (kg) \\ \midrule
            Small & 601 - 1200 \\
            Mini & 201 - 600 \\
            Micro & 11 - 200 \\
            Nano & 1.1 - 10 \\
            Pico & 0.09 - 1 \\
            Femto & 0.01 - 0.1 \\ \bottomrule
            \end{tabular}
            \end{table}

            Maleni sateliti \engl{Small} teže manje od 1200 kilograma, cijena im je nešto manje od 30 milijuna GBP (britanskih funti) za izradu i lansiranje te je potrebno od 2 do 3 godine za njihov razvoj. Mini sateliti teže do 600 kilograma i njihov razvoj traje dvije godine po cijeni do 30 milijuna GBP. Micro sateliti su teški manje od 200 kilograma i koštaju manje od 10 milijuna GBP. Ispod te klase nalaze se Nano sateliti, težine do 10 kilograma, te Pico i Femto sateliti težine manje od 1 kilograma \cite{hrvatskiVojnik}.

            \subsubsection{Cubesat format}{
                %- sto su Nanosateliti
                %    - za sto se koriste
                %- sto je cubesat satelit
                %- od cega se sastoji, 
                %    - građa, 
                %    - nacin slaganja dodatnih modula
                %- kojih sve verzije cubesat satelita postoje
                %- zasto bas kocka
                %- prednosti i mane cubesat satelita
                %- navesti negdje brojke o udijelu cubesat sateltia u ostalim satelitima kroz godine
                %- zasto je popularan cubesat format
            }
        }
        
        \subsection{Metode lansiranja}{
            %- objasniti kako se sateliti lansiraju
            %    - objasni da oni nisu glavni payload
            %    - objasni kako su pakirani
            %    - objasni u koje orbite mogu ici
            %    - objasni tvrtke koje se bave lansiranjem
            %- navedi koliko najcesce kosta lansiranje
            %- navedi kako najlakse iznajmit lansiranje
            %- rast lansiranja nanosateltia iz godine u godinu
            %    - pronaci negdje neke brojke o lansiranju cubesat satelita po godinama
        }
        
        \subsection{Povijesni pregled misija}{
            %- navesti neke komercijalne primjene do sada
            %- navesti znanstvene misije do sada
        }
    }

    \section{Korisni teret satelita}{
        \subsection{Vrste tereta}{
            %- navesti korisne terete koje satelit ima u ovisnosti o tipu misije
            %- navesti kako pravilna funkcionalnost korisnog tereta ovisi o orijentaciji  
            %- navesti neke korisne terete/sustave koji ovise o orijentaciji satelita
            %- u ovisnosti o teretu, objasniti koje vrste orijentacije koristimo (npr antena prati zemlju, itd)
        }
    
        \subsection{Važnost kontrolirane orijentacije}{
            %- navesti kraticu adcs
            %- ukratko objasniti kako estimiramo orijentaciju
            %    - algoritmi
            %    - senzori
            %- ukratko objasniti kako kontroliramo orijentaciju
            %    - algoritmi
            %    - aktuatori
            %- zasto je on jedan od najvaznijih sustava satelita
            %- sto se desava ako sustav prestane raditi ili radi nenormalno
            %    - navesti neke primjere
        }
    }
}

\chapter{Osnovni kinematički model satelita}{
    \section{Matematički zapis modela}{
        %- objasniti sve parametre kinematickog modela satelita
        %- objasniti pojednostavljenja
        %- objasniti određivanje parametara
    }
    
    \section{Orijentacija satelita}{   
        \subsection{Vrste reprezentacija}{
            \subsubsection{Eulerovi kutovi}{

            }

            \subsubsection{Kvaternioni}{

            }

            \subsubsection{Usporedba vrsta reprezentacija}{
    
            }
        }
    }
    
    \section{Kutna brzina satelita}{
        %- sto je kutna brzina satelita
        %- kako ju reprezentiramo
        %    - promjena eulerovih kuteva
        %    - promjena kvaterniona
    }
}
    
\chapter{Sustav za određivanje i upravljanje orijentacijom satelita (ADCS)}{
    \section{Uvod}{
        %- sto je sustav za određivanje i upravljanje orijentacijom satelita
        %    - navesti englesku skracenicu (koristi eng u latexu, vidi dokumentaciju pri dnu)
        %- koja je zadaca sustava za određivanje a koja sustava za upravljanje orijentacijom satelita
    }

    \section{Osnovni dijelovi}{
        \subsection{Računalo}{
            %- prikupljanje podataka
            %- obrada podataka
            %- kontrola orijentacije
            %- komunikacija
        }

        \subsection{Senzori}{
            %- koji sve postoje
            %- na kojem principima rade
            %- navesti neke tehnicke podatke svakog
            %- napraviti usporedbu između njih
            %    - cijena
            %    - preciznost
            %    - uvijeti rada
            %    - dostupnost
            %    - jednostavnost
            %    - potrosnja
        }

        \subsection{Aktuatori}{
            %- koji sve postoje
            %- na kojim principima rade
            %- tehnicki podatci
            %- usporedba između njih
            %    - cijena
            %    - preciznost
            %    - uvijeti rada
            %    - dostupnost
            %    - jednostavnost
            %    - potrosnja
        }
    }

    \section{Određivanje orijentacije}{
        \subsection{Korišteni senzori}{
            %- kako mjerimo kutnu brzinu
            %    - ziroskop
            %    - sto je ziroskop
            %    - problemi ziroskopa
        }

        \subsection{Algoritmi}{
            %- kako mozemo estimirati orijentaciju satelita
            %    - senzor fusion
            %    - navesti neke od algoritama
            %    - rekurzivni
            %    - ne rekurzivni
            %    - vidi s josipom jos neke koje je on razvijao
            %    - karla je jedan razvijala (stavi referencu)
        }
    }

    \section{Upravljanje orijentacijom}{
        \subsection{Korišteni aktuatori}{
            %- navesti aktuatore
            %- zamasnjake, magnetorqere
            %- navesti problem zasicenja zamasnjaka
        }

        \subsection{Algoritmi}{
            %- ukratko o upravljanju orijentacijom
            %- ukratko o upravljanju kutnom brzinom
            %- regulacija
            %    - navesti PID regulator kao regulator
            %    - opisati sto je PID regulator
            %    - navesti quaternion regulator kao regulator
            %        - jednadzbe
            %        - dodaj referencu na paper
        }
    }

    \section{Sklopovlje}{
        \subsection{Tiskana pločica}{
            %- ukratko o razvijenoj plocici
            %- staviti kakvu shemu
        }

        \subsection{Senzori}{
            %- zasto koristimo IMU
            %- napisati da akcelerometar ne moze raditi u svemiru
            %- mane ziroskopa
            %    - bias
            %- mane akcelerometra
            %    - sum
            %- magnetometar
            %    - model magnetskog polja zemlje
            %        - mijenja se s vremenom
        }

        \subsection{Aktuatori}{
            %- zasto koristimo bas (stavi ime) aktuator 
            %    - a ne magnetorqer npr
            %    - za magnetorqer navesti referencu na izracun parametara koji je radio Ivan Indir
            %    - navesti karlin rad, pitaj ju sto je ona radila pa ju citiraj
        }

        \subsection{Komunikacija}{
            %- zasto koristimo bluetooth komunikaciju
            %- navest ime modula
            %- kako radi modul
            %- kako komuniciramo s njime
        }
    }
    
    \section{Programska podrška}{
        \subsection{Ugradbeno računalo}{
            %- koje ugradbeno racunalo koristimo
            %- zasto smo bas to odabrali
        }

        \subsection{Organizacija projekta}{
            %- objasniti organizaciju koda
        }
        
        \subsection{Korištene biblioteke}{
            %- koje biblioteke koristimo i zasto
        }

        \subsection{Operacijski sustav}{
            %- opisati da koristimo FreeRTOS
            %- opisati koje sve dretve imamo
            %    - za svaku dretvu napisati dijagram toka
            %    - opisati funkciju dretve
            %    - opisati vrijeme izvođenja dretvi
            %    - opisati na koji nacin dretve međusobno komuniciraju
        }

        \subsection{Matematički zapis orijentacije}{
            %- jednadzbe koje koristimo u SW
        }
    
        \subsection{Izabrana metoda određivanja orijentacije}{
            %- objasniti na koji nacin radimo senzor fuzion
            %    - optimal request
            %        - napisati nesto kratko o OR algoritmu
            %        - navesti referencu na svu popratnu dokumentaciju
        }
    
        \subsection{Izabrana metoda upravljanja orijentacijom}{
            %- odabrana metoda upravljanja orijentacijom
            %    - spomeni quaternion regulator
            %- odabrana metoda upravljanja kutnom brzinom
            %    - spomeni zamasnjak
            %    - pwm
            %    - spomeni PID regulator
            %    - PID regulator za kontrolu orijentacije satelita
            %        - opisati kako ga mi koristimo
            %        - ovo isto pokupi od Arduino projekta
        }

        \subsection{Razvoj}{
            %- objasniti koji compiler koristimo
            %- objasniti koji programski jezik koristimo
            %- objasniti da je cijeli kod na githubu
            %- broj linija koda
            %    - vidi kako to dobiti u linux-u sa nekom komandom
            %- objasniti koje sve alate koristimo prilikom razvoja SW
            %    - clang format i slicno
            %    - vscode
            %    - openocd
            %- objasniti kako debugiramo
            %- slikati graf razvoja programskog koda kroz vrijeme
        }
    }
}

\chapter{Eksperimentalna verifikacija ADCS sustava}{
    \section{Opis sustava}{
        %- opisati zracni lezaj
        %- opisati kuglu
        %- opisati kako satelit stoji u kugli
        %- navesti reference za zracne lezaje
        %- parser
        %- program za iscrtavanje
        %- bluetooth modul za komunikaciju
        %- iscrtavanje orijentacije papirnatog avioncica u Octave programu
    }

    \section{Određivanje parametara}{
        \subsection{Kinematički model}{
            %- objasni kako smo dobili parametre
        }
    
        \subsection{PID regulator}{
            %- vidi cubesat na arduinu
            %- objasni kako smo dobili parametre PID regulatora
        }
    }

    \section{Računanje razlike trenutne i željene orijentacije}{
        \subsection{Eulerovi kutovi}{
            %- staviti formulu iz matlaba
        }

        \subsection{Kvaternioni}{
            %- staviti formule iz onog papera od josipa
        }
    }

    \section{Rezultati verifikacije}{
        %- ovo nezz
    }
}

% TODO: dodaj zakljucak
\chapter{Zaključak}{
    Ovdje ide zaključak.
}

\bibliographystyle{templates/template}
\bibliography{literatura}

% TODO: dodaj sazetak
\begin{sazetak}{
    Ovdje ide sažetak na hrvatskom jeziku.
}

% TODO: dodaj kljucne rijeci
\kljucnerijeci{Ovdje, idu, ključne, riječi, odvojene, zarezima.}
\end{sazetak}

% TODO: Dodaj naslov na engleskom
\engtitle{Put here title on english}
% TODO: Navedite Abstract
\begin{abstract}{
    Add abstract here.
}

% TODO: Navedite Keywords
\keywords{Add, keywords, here.}
\end{abstract}

\end{document}
