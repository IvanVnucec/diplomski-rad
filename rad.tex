\documentclass[times, utf8, diplomski, numeric]{templates/template}
\usepackage{booktabs}
\usepackage{listingsutf8}
\usepackage{xcolor}

% konfiguracija za pseudokod
\definecolor{codegreen}{rgb}{0,0.6,0}
\definecolor{codegray}{rgb}{0.5,0.5,0.5}
\definecolor{codepurple}{rgb}{0.58,0,0.82}
\definecolor{backcolour}{rgb}{0.95,0.95,0.92}

\lstdefinestyle{mystyle}{
    backgroundcolor=\color{backcolour},   
    commentstyle=\color{codegreen},
    keywordstyle=\color{magenta},
    numberstyle=\tiny\color{codegray},
    stringstyle=\color{codepurple},
    basicstyle=\ttfamily\footnotesize,
    breakatwhitespace=false,         
    breaklines=true,                 
    captionpos=b,                    
    keepspaces=true,                 
    numbers=left,                    
    numbersep=5pt,                  
    showspaces=false,                
    showstringspaces=false,
    showtabs=false,                  
    tabsize=2,
    inputencoding = utf8,  % Input encoding
    extendedchars = true,  % Extended ASCII
    literate      =        % Support additional characters
      {ž}{{\v z}}1 {č}{{\v c}}1 {š}{{\v s}}1  
      % ¿ and ¡ are not correctly displayed if inconsolata font is used
      % together with the lstlisting environment. Consider typing code in
      % external files and using \lstinputlisting to display them instead.  
}

\lstset{style=mystyle}

\begin{document}

\sveuciliste{SVEUČILIŠTE U ZAGREBU}
\fakultet{FAKULTET ELEKTROTEHNIKE I RAČUNARSTVA}

\title{Programska podrška sustava za određivanje i upravljanje orijentacijom satelita}

\thesisnumber{2346}

\author{Ivan Vnučec}

\maketitle

% Ispis stranice s napomenom o umetanju izvornika rada. 
\izvornik

% TODO: Dodaj zahvalu
\zahvala{Posebno hvala mentoru Josipu na razumijevanju, nesebičnoj pomoći i susretljivosti. Da nije bilo tebe Josipe, radio bi Web development. \\
Jednako tako hvala i Karli, bez njenog truda moj softver ne bi imao hardver.}

\tableofcontents

\chapter{Uvod}{
    \section{Mali sateliti}{
        %- sto su mali sateliti
        Mali sateliti su skupina satelita koji su, u usporedbi s konvencionalnim satelitima, uvelike ograničeni prema masi, cijeni, kao i prema vremenu potrebnom da ih se razvije. Sve je to izravan rezultat revolucije u elektronici i računarstvu.\cite{hrvatskiVojnik}.

        %- tko ih koristi
        Primjene malih satelita nalazimo u svim granama svemirske industrije: u telekomunikaciji, navigaciji, opservaciji Zemlje, znanstvenim istraživanjima i dr.
        
        %- zasto ih koristimo
        Njihova zastupljenost na tržištu raste iz godine u godinu jer pružaju jeftin i brz razvoj uz relativno male tehničke zahtjeve \cite{rastMalihSatelita}. Primjenom većeg broja malih satelita povezanih u konstelaciju, moguće je primjerice pokriti veću površinu zemlje i tako stvoriti uvijek dostupni internet ili navigacijski sustav.
        
        \subsection{Podjela}{
            Iako postoje više vrsta podjela satelita u ovisnosti od izvora do izvora, navest ćemo podjelu prema FAA \engl{Federal Aviation Administration} koja uključuje satelite s najvećom, pa sve do satelita s najmanjom masom \cite{podjelaPremaMasi}. Podjelu je moguće vidjeti u tablici \ref{tbl:podjelaSatelita}.

            \begin{table}[htb]
            \caption{Podjela malih satelita prema masi}
            \label{tbl:podjelaSatelita}
            \centering
            \begin{tabular}{ll} \toprule
            Naziv kategorije satelita & Masa (kg) \\ \midrule
            Small & 601 - 1200 \\
            Mini & 201 - 600 \\
            Micro & 11 - 200 \\
            Nano & 1.1 - 10 \\
            Pico & 0.09 - 1 \\
            Femto & 0.01 - 0.1 \\ \bottomrule
            \end{tabular}
            \end{table}

            Maleni sateliti \engl{Small} teže manje od 1200 kilograma, cijena im je nešto manje od 30 milijuna GBP (britanskih funti) za izradu i lansiranje te je potrebno od 2 do 3 godine za njihov razvoj. Mini sateliti teže do 600 kilograma i njihov razvoj traje dvije godine po cijeni do 30 milijuna GBP. Micro sateliti su teški manje od 200 kilograma i koštaju manje od 10 milijuna GBP. Ispod te klase nalaze se Nano sateliti, težine do 10 kilograma, te Pico i Femto sateliti težine manje od 1 kilograma \cite{hrvatskiVojnik}. Popularnost malih satelita iz godine u godinu raste što prikazuje slika \ref{fig:lansiranja_po_godini}.

            \begin{figure}[htb]
            \centering
            \includegraphics[width=1.0\textwidth]{images/lansiranja_po_godini.png}
            \caption{Broj lansiranja različitih tipova malih satelita kroz godine \cite{nanosats}.}
            \label{fig:lansiranja_po_godini}
            \end{figure}
        }

            \subsubsection{Cubesat format}{
                Cubesat format osmišljen je 1999. godine suradnjom Cal Poly sveučilišta u San Luis Obispu i sveučilište Stanford. Projekt je imao za zadatak standardizirati Cubesat format i definirati njegove specifikacije. Cubesat format definiramo kao kocku veličine 10 cm x 10 cm x 10 cm (1U) \cite{fersat}. Danas se mogu kombinirati tri (3U), šest (6U) i više takvih blokova. 1U, 2U i 3U Cubesat formati prikazani su na slici \ref{fig:cubesat_format}.

                \begin{figure}[htb]
                \centering
                \includegraphics[width=0.7\textwidth]{images/cubesat_format.jpg}
                \caption{Prikaz 1U, 2U i 3U Cubesat formata.}
                \label{fig:cubesat_format}
                \end{figure}

                Nakon definicije Cubesat formata, 1U format postao je najpopularniji format za razvoj i lansiranje i kojim su se mogle ostvariti jednostavne funkcije, kao što su, na primjer: slikanje Zemlje i prijenos podataka na Zemlju te niz jednostavnijih mjerenja koje se trebaju prenijeti na zemlju \cite{fersat}.
            }
        }

        \subsection{Metode lansiranja}{
            Mali sateliti lansiraju se kao sekundarni teret i to u posebno izrađenim kavezima u kojima stane više malih satelita istovremeno. Za izbacivanje u svemir koristi se opruga, jer Cubesat sateliti nemaju propulziju. Nakon što je izdan izbacivački signal, mehanička opruga vrši silu na male satelite koji potom klizući šinama izlaze kroz vrata kaveza \cite{fersat}. Prikaz kaveza možemo vidjeti na slici \ref{fig:kavez}.

            \begin{figure}[htb]
            \centering
            \includegraphics[width=0.4\textwidth]{images/kavez.jpg}
            \caption{Lansirni kavez Cubesat satelita \cite{kavez_cite}.}
            \label{fig:kavez}
            \end{figure}

            Let rakete koja nosi primarni i sekundarni teret je zamišljen tako da primarni teret dovede na određenu orbitu iznad zemlje i pusti teret. Neko vrijeme nakon primarnog tereta, izbacuje se sekundarni teret (jedan ili više malih satelita). Lansiranja primarnog tereta su u današnje vrijeme vrlo česta: datumi lansiranja i orbita znaju se mjesecima unaprijed tako da korisnici sekundarnog tereta mogu planirati orbitu koju žele. Korisnici sekundarnog tereta mogu birati samo između ponuđenih orbita, ali ne mogu mijenjati orbitu ili datum lansiranja koje su odabrali. Cijene lansiranja 1U Cubesat-a se kreću oko €90.000, a za 3U Cubesat oko €200.000. Jedna raketa može u svemir ponijeti 10 ili više Cubesat satelita \cite{fersat}. 

            Poslovanje lansiranja malih satelita je organizirano kroz pružatelje lansirnih usluga \engl{Launch Services Provider - LSP} koji zakupljuju prostor za sekundarni teret od poduzeća koje lansiraju rakete \engl{Launch Logistics Provider - LLP}. Danas postoji više LSP-ova i stranka koja želi lansirati satelit ima mogućnost izbora po cijeni i kvaliteti usluge. Poduzeća koja lansiraju satelite u svemir nalaze se po cijelom svijetu \cite{fersat}. Listu nekih pružatelja usluga lansiranja malih satelita moguće je vidjeti u tablici \ref{tbl:pru_lan_uslu}.

            \begin{table}[htb]
            \caption{Lista pružatelja usluga lansiranja malih satelita \cite{fersat}.}
            \label{tbl:pru_lan_uslu}
            \centering
            \begin{tabular}{lll} 
            \toprule
            Ime & Lokacija & Web adresa \\ 
            \midrule
            Innovative Solutions in Space & Nizozemska & \url{www.isispace.nl} \\
            ECM Space & Njemačka & \url{www.ecm-space.de} \\
            UT AeroSpace Studies & Kanada & \url{www.utias-sfl.net} \\
            CubeCab (3U+) & SAD & \url{www.cubecab.com} \\
            Nano Avionics & Litva & \url{https://nanoavionics.com/} \\
            Space Flight (3U+) & SAD & \url{https://spaceflight.com/} \\
            \bottomrule
            \end{tabular}
            \end{table}

            Postupak lansiranja započinje potpisivanjem ugovora pri kojem korisnik odabire jedan od ponuđenih datuma i orbita. Pri potpisivanju ugovora korisnik plaća oko 50\% cijene usluge. Potpisivanje ugovora se događa barem dvanaest mjeseci prije planiranog lansiranja. Tijekom slijedećih dva do šest mjeseci korisnik mora obaviti sva testiranja. Šest mjeseci prije lansiranja, korisnik dobiva potvrdu o mjestu na raketi i plaća dodatnih 20\% dogovorene cijene. Dva mjeseca prije lansiranja, korisnikov satelit se predaje pružatelju lansirnih usluga koji korisnikov satelit prevozi do lokacije lansiranja. Stigavši na lansirnu lokaciju, korisnik plaća dodatnih 20\% cijene. Konačno, u trenutku lansiranja, korisnik plaća zadnjih 10\% cijene. Ukoliko lansiranje ne uspije nema povrata sredstava. Obročna plaćanja ne vrijede za sve pružatelje usluga već neki imaju drugačije oblike plaćanja. Nakon lansiranja pružatelj lansirnih usluga definira i prati orbitu svakog lansiranog satelita i predaje korisniku informacije o točnoj orbiti u kojoj se satelit nalazi \cite{fersat}. 
        
        \subsection{Povijesni pregled misija}{
            Baza podataka Nano i Cubesat satelita \cite{nanosats} sadrži preko 2000 podataka o lansiranjima Cubesat satelita od 1998. godine. Prvo lansiranje Cubesat satelita zbilo se 2003. godine iz grada Plesetska u Rusiji pod zaslugom \emph{Eurockot Launch Services's Multiple Orbit} misije. Cubesat sateliti koji su pušteni u orbitu bili su sateliti nekoliko tvrtki iz Danske, Japana, Kanada i SAD-a \cite{prva_misija}.

            Nekoliko godina kasnije, točnije 2012. godine, lansirano je 11 Cubesat satelita pomoću rakete Atlas V. To je do tada bio najveći broj istovremeno lansiranih satelita gdje je najveći bio čak 24U veličine \cite{najveci_lansirni_broj}.

            Od tada do danas lansirano je preko 1000 malih satelita što je vidljivo na slici \ref{fig:lansiranja_po_godini}.
        }
    }

    \section{Korisni teret satelita}{
        Korisni teret satelita \engl{payload} jest podsustav, ili skupina podsustava, koji imaju za cilj ispuniti zadaću koja im je dodijeljena (misija). Primjerice, ako je cilj satelita fotografiranje Zemljine površine, onda će njegov koristan teret biti sustav za fotografiranje (kamera). Ili primjerice ako želimo uspostaviti komunikaciju sa satelitom, što je gotovo uvijek slučaj, onda će njegov korisni sustav biti sustav za komunikaciju.

        \subsection{Vrste tereta}{
            Ovisno o tipu misije, razlikujemo podsustave koje je moguće svrstati u nekoliko sljedećih kategorija:

            \begin{itemize}
                \item Opservacija Zemlje
                \item Komunikacija
                \item Navigacija
                \item Znanost i tehnologija
            \end{itemize}

            Sustavi za opservaciju Zemlje sadrže uređaje koji promatraju Zemljinu površinu ili atmosferu. Najčešće takav sustav sadrži kameru koja fotografira u odabranom dijelu spektra. Zbog toga što na takvom sustavu osim kamere imamo i objektive koji omogućuju velika uvećanja, potrebno je da posjedujemo mogućnost preciznog usmjeravanja kamere. Također, sustav za komunikaciju, osim sklopovlja za obradu signala, sadrži i antene. Kako bi pospješili prijem i odašiljanje, potrebno je usmjeriti antenu prema drugoj anteni na Zemlji \cite{sattelitePayload}.

            \subsection{Važnost kontrolirane orijentacije}{
                Osim gore opisanih sustava kojima je važna kontrola orijentacije, navest ćemo još jedan primjer gdje nam je kontrola orijentacije važna. Naime, prilikom izbacivanja satelita iz samog prijevoznog sredstva u svemir, događa se nekontrolirana rotacija satelita. Kako bi satelit zaustavio rotaciju, aktivni sustav kontrolu orijentacije ulazi u posebni način rada \engl{detumbling} u kojem će zaustaviti nekontroliranu rotaciju. Takav zahvat može potrajati tjednima ili mjesecima, ovisno o tipu satelita \cite{fersat}.
            }
        }
    }
}

\chapter{Opis orijentacije satelita}{
    Kako bi matematički prikazali rotacijsko gibanje satelita, satelit je prvo potrebno modelirati kao kruto tijelo. Mi ćemo u nastavku navesti samo osnovne izraze potrebne za modeliranje satelita, a potpune matematičke izraze i izvode moguće je pronaći u popratnoj literaturi \cite{adcsKnjiga}.

    Ako govorimo o orijentaciji, važno je objasniti pojam referentnog sustava \engl{reference plane}. Referentni sustav je skup 3 međusobno ortogonalnih vektora prema kojima određujemo sve ostale vektore, bilo njihovu orijentaciju, bilo pomak. Jedan možda najpoznatiji primjer takvog referentnog sustava je Kartezijev referentni sustav gdje smo kao referentne vektore izabrali 3 jedinična vektora $\boldsymbol{i}$, $\boldsymbol{j}$ i $\boldsymbol{k}$ prema kojima onda određujemo sve ostale vektore koje definiramo u Kartezijevom prostoru. 

    Moguće je imati više referentnih sustava. Prelazak iz jednog u drugi referentni sustav moguće je ostvariti transformacijama s rotacijskim matricama o kojima ćemo govoriti kasnije. Referentne sustave najčešće označavamo kao npr. $\mathcal{F}_a$ ili $\mathcal{F}_b$. Rotacijska matrica koja bi određivala orijentaciju $\mathcal{F}_a$ referentnog sustava naspram $\mathcal{F}_b$ referentnog sustava označavamo kao $\boldsymbol{C}_{ab}$.

    Da bi definirali orijentaciju satelita, prvo moramo definirati referentne sustave. Prvi referentni sustav kojeg definiramo je inercijski referentni sustav $\mathcal{F}_G$. U klasičnoj fizici, inercijski sustav označava referentni sustav koji ne posjeduje akceleraciju \cite{inertialFrame}. Drugi referentni sustav kojeg definiramo je tzv. \emph{body} referentni sustav $\mathcal{F}_b$ koji je nepomičan s obzirom na gibanje čvrstog tijela (satelita).

    Orijentacija satelita u referentnom sustavu $\mathcal{F}_b$ naspram inercijskog referentnog sustava $\mathcal{F}_G$ označavamo rotacijskom matricom $\boldsymbol{C}_{bG}$.

    \section{Osnovni parametri satelita}{
    \label{section:osnovni_parametri_satelita}
        Inercijska matrica \engl{Moment of inertia tensor} u pojednostavljenom smislu označava moment tromosti krutog tijela oko pojedine osi rotacije. Naglašavamo da je u pitanju pojednostavljena definicija i da ona nije strogo točna, no za naše potrebe je ona dovoljna. Inercijsku matricu predstavljamo kao:

        \begin{equation}
        \boldsymbol{J} = 
        \begin{bmatrix}
            J_{xx} & J_{xy} & J_{xz} \\
            J_{yx} & J_{yy} & J_{yz} \\
            J_{zx} & J_{zy} & J_{zz}
        \end{bmatrix}
        .
        \end{equation}

        Moguće je pokazati da za svako kruto tijelo možemo pojednostaviti Inercijsku matricu te tako dobiti dijagonalnu inercijsku matricu. Takva matrica nam omogućuje izvedbu jednostavnijih izraza za rotaciju krutog tijela. Zbog nepotrebnog ulaženja u detalje kako je moguće dobiti dijagonalnu matricu, navest ćemo samo izraz kao:

        \begin{equation}
        \boldsymbol{I} = 
        \begin{bmatrix}
            I_{xx} & 0      & 0 \\
            0      & I_{yy} & 0 \\
            0      & 0      & I_{zz} \\
        \end{bmatrix}
        .
        \end{equation}

        Radi jednostavnosti, pretpostavit ćemo da model kutne brzine satelita u Laplaceovoj domeni izgleda kao sustav prvog reda:
        
        \begin{equation}
        \label{eq:laplace_prvi_red}
        H(s) = \frac{Y(s)}{X(s)} = K \frac{1/\tau}{s + 1/\tau},
        \end{equation}

okarakterizirati
        gdje su $K$ i $\tau$ koeficijenti prijenosne funkcije. Parametar $K$ mogli bi okarakterizirati kao koeficijent pretvorbe mjernih jedinica ulaza $X(s)$ i izlaza $Y(s)$, a parametar $\tau$ kao koeficijent tromosti sustava (vremenska ovisnost).
    }

    \section{Eulerovi kutovi}{
        Eulerovi kutovi predstavljaju 3 kuta između nekog vektora i ortogonalnih osi koje razapinju Kartezijev koordinatni sustav. Ako želimo predstaviti rotaciju između jednog proizvoljnog vektora i nekog drugog vektora, onda to možemo napraviti na nekoliko načina. Jedan primjer takve rotacije navest ćemo u sljedećem primjeru:

        \begin{enumerate}
            \item Rotacija za kut $\psi$ oko originalne z-osi \engl{yaw}.
            \item Rotacija za kut $\theta$ oko tranzicijske y-osi \engl{pitch}.
            \item Rotacija za kut $\phi$ oko transformirane x-osi \engl{roll}.
        \end{enumerate}

        Uzmemo li za primjer 3 uzastopne rotacije: prve oko originalne z-osi, zatim druge oko tranzicijske y-osi i na kraju treće oko transformirane x-osi dobivamo rotaciju prikazanu na slici ispod: 

        \begin{figure}[htb]
        \centering
        \includegraphics[width=1.0\textwidth]{images/eulerovi_kutovi.jpg}
        \caption{Tri uzastopne rotacije oko z-osi, y-osi i x-osi za kut $\psi$, $\theta$, i $\phi$}
        \label{fig:euler_rotacija}
        \end{figure}
    }

    \section{Kvaternioni}{
        Kvaternion je četvero-dimenzionalni vektor s kojim je moguće opisati orijentaciju tijela. Kvaternion u usporedbi s Eulerovim kutovima posjeduje dvije prednosti (o kojima nešto više u nastavku): prva je da ne posjeduje singularnost, i druga da matematičke relacije modeliranja orijentacije više nisu trigonometrijske već algebarske prirode, što ih dodatno čini pogodnima za računanje.

        Kvaternion opisuje dvije stvari potrebne za opis rotacije: glavnu os rotacije \engl{principal axis} $\overrightarrow{\boldsymbol{a}}$ i kuta rotacije $\phi$ kao što je vidljivo na slici \ref{fig:principal_axis_rotation}. Zbog toga, razlikujemo vektorski i skalarni dio koje definiramo kao:

        \begin{figure}[htb]
        \centering
        \includegraphics[width=0.2\textwidth]{images/principal_axis_rotation.png}
        % TODO: dodaj strelice iznad vektora (oprez: latex zeza nesto)
        \caption{Prikaz rotacije vektora $\boldsymbol{b}$ oko glavne osi rotacije $\boldsymbol{a}$ za kut $\phi$.}
        \label{fig:principal_axis_rotation}
        \end{figure}

        \begin{equation}
        \label{eq:kvaternion}
        \begin{array}{rcl}
            \boldsymbol\epsilon &  = & \overrightarrow{\boldsymbol{a}}\sin\frac{\phi}{2} = \left[\epsilon_{x} \; \epsilon_{y} \; \epsilon_{z} \right], \\
            \eta & = & \cos\frac{\phi}{2}
        \end{array}
        \end{equation}

        U praksi postoje dvije interpretacije kvaterniona u ovisnosti o tome ide li skalarni dio kao prvi element kvaterniona ili kao posljednji. U našem radu pretpostavit ćemo oblik kvaterniona gdje je skalarni dio prvi, a vektorski drugi:

        \begin{equation}
        \label{eq:kvaternion_elem}
            \boldsymbol{q}=
            \left[\eta \; \boldsymbol\epsilon \right] = \left[q_{1} \; q_{2} \; q_{3} \; q_{4}\right]
        \end{equation}
    }

    \section{Rotacijska matrica}{
        Rotacijska matrica opisuje transformaciju vektora iz originalnog u rotirani, za neku poznatu rotaciju. U nastavku ćemo navesti rotacijske matrice koje određujemo pomoću Eulerovih kutova i pomoću kvaterniona i dati ćemo primjer rotacije vektora.

        \subsection{Prikaz pomoću Eulerovih kutova}{
            U nastavku ćemo navesti prikaze rotacijskih matrica pomoću Eulerovih kutova. Prvo ćemo navesti kako izgledaju rotacijske matrice kada imamo rotacije samo oko glavnih osi, a zatim ćemo navesti izraz za rotacijsku matricu za 3-2-1 uzastopnu rotacijsku sekvencu.

            \subsubsection{Rotacije oko glavnih osi}{
                Bez ulaženja u detalje, navest ćemo redom kako izgledaju rotacijske matrice za rotacije oko z, y i x osi. Prikaze tih rotacija moguće je vidjeti na slici \ref{fig:principal_rotations}.

                Rotacija oko z-osi:

                \begin{equation}
                \label{eq:principal_rot_z}
                \boldsymbol{C}_{z}(\theta_{z}) = 
                \begin{bmatrix}
                    \cos\theta_{z}   & \sin\theta_{z}    &  0 \\
                    -\sin\theta_{z}  & \cos\theta_{z}    &  0 \\
                    0                & 0                 &  1
                \end{bmatrix}
                .
                \end{equation}

                Rotacija oko y-osi:

                \begin{equation}
                \label{eq:principal_rot_y}
                \boldsymbol{C}_{y}(\theta_{y}) = 
                \begin{bmatrix}
                    \cos\theta_{y}   &    0    &  -\sin\theta_{y} \\
                    0                &    1    &  0               \\
                    \sin\theta_{y}   &    0    &  \cos\theta_{y}
                \end{bmatrix}
                .
                \end{equation}

                Rotacija oko x-osi:

                \begin{equation}
                \label{eq:principal_rot_x}
                \boldsymbol{C}_{x}(\theta_{x}) = 
                \begin{bmatrix}
                    1    &      0           &  0 \\
                    0    & \cos\theta_{x}   &  \sin\theta_{x} \\
                    0    & -\sin\theta_{x}  &  \cos\theta_{x}
                \end{bmatrix}
                .
                \end{equation}

                \begin{figure}[htb]
                \centering
                \includegraphics[width=1.0\textwidth]{images/principal_rotations.png}
                \caption{Prikaz rotacija oko glavnih osi.}
                \label{fig:principal_rotations}
                \end{figure}
            }

            \subsubsection{3-2-1 rotacijska sekvenca}{
                Rotacija sekvenca koja je jako česta u zrakoplovno-svemirskoj industriji označava se kao 3-2-1 orijentacijska sekvenca \engl{attitude sequence}. Takvo ime je uzeto zato što je prva rotacija oko originalne z-osi (oznaka 3), zatim slijedi rotacija oko tranzicijske y-osi (oznaka 2) te na kraju rotacija oko transformirane x-osi (oznaka 1).

                3-2-1 rotacijsku matricu moguće je prikazati pomoću Eulerovih kutova kao:

                \begin{equation}
                \label{eq:euler_rot_mat}
                \begin{array}{rcl}
                \boldsymbol{C}_{21}(\phi, \theta, \psi) & = & \boldsymbol{C}_{x}(\phi) \boldsymbol{C}_{y}(\theta) \boldsymbol{C}_{z}(\psi) \\
                & = &
                \begin{bmatrix}
                    c_{\theta}c_{\psi}                            & c_{\theta}s_{\psi}                            & -s_{\theta} \\
                    s_{\phi}s_{\theta}c_{\psi} - c_{\phi}s_{\psi} & s_{\phi}s_{\theta}s_{\psi} + c_{\phi}c_{\psi} & s_{\phi}c_{\theta} \\
                    c_{\phi}s_{\theta}c_{\psi} + s_{\phi}s_{\psi} & c_{\phi}s_{\theta}s_{\psi} - s_{\phi}c_{\psi} & c_{\phi}c_{\theta}
                \end{bmatrix}
                \end{array}
                ,
                \end{equation}

                gdje radi preglednosti vrijedi $s_{b}=\sin(b)$ i $c_{b}=\sin(b)$.

                Primijetimo kako smo 3-2-1 rotacijsku matricu dobili tako što smo množili pojedinačne matrice rotacija oko glavnih osi (vidi jednadžbe \ref{eq:principal_rot_z}, \ref{eq:principal_rot_y} i \ref{eq:principal_rot_x}). Važno je napomenuti kako u ovom slučaju ne vrijedi komutativnost i da je redoslijed umnoška matrica bitan.
            }

            \subsubsection{Singularnost Eulerovih kutova}{
            \label{subsubsection:singularnost_eulerovih_kutova}
                Jedna mana reprezentacije rotacije pomoću Eulerovih kutova je tzv. singularnost. Može se pokazati da za svaku parametrizaciju rotacije može doći do singulariteta. Za naš primjer 3-2-1 sekvence singularitet nastaje kada je $\theta=\pm90^{\circ}$. Za takav slučaj rotacijska matrica postaje:

                \begin{equation}
                \boldsymbol{C}_{21}(\phi, 90^{\circ}, \psi) =
                \begin{bmatrix}
                    0                 & 0                  & -1 \\
                    \sin(\phi - \psi) & \cos(\phi - \psi)  &  0 \\
                    \cos(\phi - \psi) & -\sin(\phi - \psi) &  0
                \end{bmatrix}
                .
                \end{equation}

                Fizikalno, zbog singulariteta prva i treća rotacija unutar 3-2-1 sekvence se odvijaju oko jedne te iste osi. U tom slučaju roll i yaw ($\phi$ i $\psi$) su zapravo jednaki kutovi i ne mogu se jednoznačno odrediti. Izvan singulariteta, moguće je jednoznačno odrediti sva tri Eulerova kuta. 
            }

            \subsubsection{Određivanje Eulerovih kutova iz rotacijske matrice}{
                Ako rotacijsku matricu označimo kao:

                \begin{equation}
                \label{eq:rotation_matrix}
                \boldsymbol{C} =
                \begin{bmatrix}
                    c_{11} & c_{12} & c_{13} \\
                    c_{21} & c_{22} & c_{23} \\
                    c_{31} & c_{32} & c_{33} \\
                \end{bmatrix}
                ,
                \end{equation}

                iz jednadžbe \ref{eq:euler_rot_mat} moguće je odrediti Eulerove kutove kao:

                \begin{equation}
                \begin{array}{rcl}
                    \phi   & = & \tan^{-1}(c_{23}/c_{33}),\\
                    \theta & = & -\sin^{-1}(c_{13}),\\
                    \psi   & = & \tan^{-1}(c_{12}/c_{11}).
                \end{array}
                \end{equation}
            }

            \textbf{Naglašavamo kako je iznimno važno da prilikom definiranja rotacije pomoću Eulerovih kutova naglasimo o kojoj sekvenci rotacije se radi (npr. 3-2-1) jer rotacijske sekvence ne komutiraju!}
        }

        \subsection{Prikaz pomoću kvaterniona}{
            Rotacijsku matricu pomoću kvaterniona definiranog ranije u radu označavamo kao:

            \begin{equation}
                \boldsymbol{C} = (2\eta^2-1)\boldsymbol{1} + 2\boldsymbol\epsilon\boldsymbol\epsilon^{T}-2\eta\boldsymbol\epsilon^{\times},
            \label{eq:quat_rot_matrix}
            \end{equation}

            gdje je $\boldsymbol\epsilon^{\times}$ definiran kao:

            \begin{equation}
                \boldsymbol\epsilon^{\times} =
                \begin{bmatrix}
                    0 & -\epsilon_{z} & \epsilon_{y} \\
                    \epsilon_{z} & 0 & -\epsilon_{x} \\
                    -\epsilon_{y} & \epsilon_{x} & 0 \\
                \end{bmatrix}
                .
            \end{equation}

            Napominjemo da prilikom računanja rotacijske matrice kvaternion $\boldsymbol{q}$ prvo normaliziramo na jediničnu duljinu jer tako garantiramo da množenje rotacijske matrice i vektora neće rezultirati promjenom duljine vektora, već samo promjenu smjera, a to je ono što želimo. Kvaternion normaliziramo na sljedeći način:

            \begin{equation}
                \boldsymbol{q}_{norm} = \frac{\boldsymbol{q}}{||\boldsymbol{q}||} = \frac{\boldsymbol{q}}{\sqrt{q_1^2 + q_2^2 + q_3^2 + q_4^2}}
            \end{equation}

            Ako li raspišemo rotacijsku matricu definiranu pomoću kvaterniona u jednadžbi \ref{eq:kvaternion_elem} dobivamo izraz \cite{uvod_u_svemirske} koji je pogodniji za računanje:

            \begin{equation}
                \boldsymbol{C} =
                \begin{bmatrix}
                    q_1^2 + q_2^2 - q_3^2 - q_4^2 & 2(q_2q_3 + q_1q_4) & 2(q_2q_4 - q_1q_3) \\
                    2(q_2q_3 - q_1q_4) & q_1^2 - q_2^2 + q_3^2 - q_4^2 & 2(q_3q_4 + q_1q_2) \\
                    2(q_2q_4 + q_1q_3) & 2(q_3q_4 - q_1q_2) & q_1^2 - q_2^2 - q_3^2 + q_4^2 
                \end{bmatrix}
            \label{eq:rot_mat_quat}
            \end{equation}

            \subsubsection{Određivanje kvaterniona iz rotacijske matrice}{
                Pomoću rotacijske matrice definirane u jednadžbi \ref{eq:rotation_matrix} moguće je odrediti sva 4 člana kvaterniona tako da prvo odredimo skalarni član kao:

                \begin{equation}
                    \eta = \pm\frac{(\text{trace}[\boldsymbol{C}] + 1)^{\frac{1}{2}}}{2}.
                \end{equation}

                Predznak skalarnog člana nije bitan jer on fizikalno samo označava smjer rotacije. Pozitivan ili negativan predznak daju istu rotaciju.

                Jednom kada smo odredili $\eta$, moguće je odrediti $\boldsymbol\epsilon$. Može se pokazati da za $\eta \neq 0$ vrijedi:

                \begin{equation}
                \begin{array}{rcl}
                    \epsilon_{x} & = & \frac{c_{23} - c_{32}}{4\eta}, \\
                    \epsilon_{y} & = & \frac{c_{31} - c_{13}}{4\eta}, \\
                    \epsilon_{z} & = & \frac{c_{12} - c_{21}}{4\eta}.
                \end{array}
                \end{equation}

                Za slučaj kada je $\eta=0$, vrijedi da je kut zakretanja jednak $\phi=\pm\pi$. Da se pokazati da je moguće izračunati elemente $\boldsymbol\epsilon$ vektora kao:

                \begin{equation}
                \begin{array}{rcl}
                    |\epsilon_{x}| & = & (\frac{c_{11} + 1}{2})^{\frac{1}{2}}, \\
                    |\epsilon_{y}| & = & (\frac{c_{22} + 1}{2})^{\frac{1}{2}}, \\
                    |\epsilon_{z}| & = & (\frac{c_{33} + 1}{2})^{\frac{1}{2}}.
                \end{array}
                \end{equation}

                Za odabir predznaka svakog elementa $\boldsymbol\epsilon$ vektora čitatelja se savjetuje da pogleda u izvor \cite{adcsKnjiga}.
            }
        }

        \subsection{Rotacija vektora pomoću rotacijske matrice}{
            U nastavku ćemo navesti izraze za rotaciju vektora pomoću rotacijskih matrica. Prvi primjer se odnosi na rotaciju oko z-osi, a drugi primjer se odnosi na 3-2-1 rotaciju. 

            Naglašavamo kako će rotacija biti izvedena u istom referentnom sustavu i nećemo se baviti transformacijama iz jednom u drugi koordinatni sustav. Za potrebe transformacija koordinatnih sustava čitatelj se upućuje na izvor \cite{adcsKnjiga}.

            \subsubsection{Rotacija oko z-osi}{
                Ako želimo vektor $\overrightarrow{\boldsymbol{v}}=\left[ x \; y \; z \right]^T$ rotirati oko npr. z-osi za 30 stupnjeva, prvo određujemo rotacijsku matricu pomoću jednadžbe \ref{eq:principal_rot_z} kao:

                \begin{equation}
                \boldsymbol{C}_{z}(30^{\circ}) = 
                \begin{bmatrix}
                    \cos30^{\circ}   & \sin30^{\circ}    &  0 \\
                    -\sin30^{\circ}  & \cos30^{\circ}    &  0 \\
                    0                & 0                 &  1
                \end{bmatrix}
                ,
                \end{equation}

                zatim pomnožimo vektor $\overrightarrow{\boldsymbol{v}}$ i matricu $\boldsymbol{C}_{z}$ da dobijemo rotirani vektor $\overrightarrow{\boldsymbol{v}}'$:

                \begin{equation}
                \begin{array}{rcl}
                    \overrightarrow{\boldsymbol{v}}' & = & \boldsymbol{C}_{z}(30^{\circ}) \overrightarrow{\boldsymbol{v}} \\
                    & = &
                \begin{bmatrix}
                    \cos30^{\circ}   & \sin30^{\circ}    &  0 \\
                    -\sin30^{\circ}  & \cos30^{\circ}    &  0 \\
                    0                & 0                 &  1
                \end{bmatrix}
                \begin{bmatrix}
                    x \\
                    y \\
                    z
                \end{bmatrix}
                \end{array}
                .
                \end{equation}
            }

            \subsubsection{3-2-1 rotacija}{
                Ako želimo vektor $\overrightarrow{\boldsymbol{v}}$ definiran u prošlom primjeru rotirati prvo oko z-osi za 30 stupnjeva, tranzicijske y-osi za 20 stupnjeva i na kraju oko transformirane x-osi za 10 stupnjeva, prvo određujemo rotacijsku matricu pomoću jednadžbe \ref{eq:euler_rot_mat} kao:

                \begin{equation}
                \boldsymbol{C}_{21}(10^\circ, 20^\circ, 30^\circ) = \boldsymbol{C}_{x}(10^\circ) \boldsymbol{C}_{y}(20^\circ) \boldsymbol{C}_{z}(30^\circ),
                \end{equation}

                zatim pomnožimo vektor $\overrightarrow{\boldsymbol{v}}$ i matricu $\boldsymbol{C}_{21}$ da dobijemo rotirani vektor $\overrightarrow{\boldsymbol{v}}'$:

                \begin{equation}
                    \overrightarrow{\boldsymbol{v}}' = \boldsymbol{C}_{21}(10^\circ, 20^\circ, 30^\circ) \overrightarrow{\boldsymbol{v}}.
                \end{equation}
            }
        }
    }

    \section{Jednadžba rotacijskog gibanja}{
        Ako govorimo o jednadžbi rotacije satelita, važno se prisjetiti definicija tzv. \emph{body} $\mathcal{F}_b
        $ i inercijskog referentnog sustava $\mathcal{F}_G$. Orijentaciju između ta dva referentna sustava moguće je opisati orijentacijskom matricom $\boldsymbol{C}_{bG}$ i to na barem dva načina: pomoću Eulerovih kutova ili pomoću kvaterniona kao što je opisano u prethodnim poglavljima.

        Kako je $\mathcal{F}_b$ referentni sustav nepomičan s obzirom na tijelo satelita, kutnu brzinu \emph{body} referentnog sustava $\mathcal{F}_b$ s obzirom na $\mathcal{F}_G$ inercijskog sustava označavamo kao:

        \begin{equation}
            \boldsymbol{\omega}_{bG} = \left[ \omega_1 \; \omega_2 \; \omega_2\right]^T.
        \end{equation}

        \subsection{Prikaz pomoću Eulerovih kutova}{
            Jednadžbu rotacijskog gibanja pomoću Eulerovih kutova definiramo kao umnožak 3-2-1 rotacijske sekvence i kutne brzine kao:

            \begin{equation}
                \begin{bmatrix}
                    \dot{\phi} \\
                    \dot{\theta} \\
                    \dot{\psi}
                \end{bmatrix}
                =
                \begin{bmatrix}
                    1 & \sin\phi\tan\theta & \cos\phi\tan\theta \\
                    0 & \cos\phi & -\sin\phi \\
                    0 & \sin\phi\sec\theta & \cos\phi\sec\theta
                \end{bmatrix}
                \boldsymbol{\omega}_{bG}.
            \end{equation}
        }

        \subsection{Prikaz pomoću kvaterniona}{
            Jednadžbu rotacijskog gibanja pomoću kvaterniona i kutne brzine definiramo kao:

            \begin{equation}
            \label{eq:rot_gib}
            \begin{array}{rcl}
                \dot{\boldsymbol\epsilon} &  = & \frac{1}{2}(\eta\boldsymbol{1} + \boldsymbol{\epsilon}^\times)\boldsymbol{\omega}_{bG}, \\
                & & \\
                \dot{\eta} & = & -\frac{1}{2}\boldsymbol{\epsilon}^T\boldsymbol{\omega}_{bG}.
            \end{array}
            \end{equation}

            Radi lakšeg računanja, raspišemo li jednadžbu \ref{eq:rot_gib} pomoću definicije kvaterniona u jednadžbi \ref{eq:kvaternion_elem} dobiti ćemo relaciju za rotacijsko gibanje \cite{uvod_u_svemirske} kao:

            \begin{equation}
            \label{eq:rot_qib_elem}
                \dot{\boldsymbol{q}} = \frac{1}{2} \boldsymbol{S}(\boldsymbol{\omega}_{bG}) \boldsymbol{q}
            \end{equation}

            gdje je matrica $\boldsymbol{S}(\boldsymbol{\omega}_{bG})$ definirana kao:

            \begin{equation}
                \boldsymbol{S}(\boldsymbol{\omega}_{bG}) =
                \begin{bmatrix}
                    0 & -\omega_1 & -\omega_2 & -\omega_3 \\
                    \omega_1 & 0 & \omega_3 & -\omega_2 \\
                    \omega_2 & -\omega_3 & 0 & \omega_1 \\
                    \omega_3 & \omega_2 & -\omega_1 & 0 
                \end{bmatrix}
                .
            \end{equation}
        }
    }

    \section{Usporedba Eulerovih kutova i kvaterniona}{
        Kao što smo već pisali u poglavlju \ref{subsubsection:singularnost_eulerovih_kutova}, dva su problema prikazivanja orijentacije pomoću Eulerovih kutova. Prvi problem je singularitet, gdje za neke kutove koje reprezentiraju istu orijentaciju (vidi poglavlje \ref{subsubsection:singularnost_eulerovih_kutova}) nije moguće jednoznačno odrediti Eulerove kutove iz rotacijske matrice. Drugi problem je što za opisivanje orijentacije koristimo trigonometrijske funkcije koje je teže izračunati na računalima nego primjerice algebarske.

        Prikaz orijentacije kvaternionom rješava nas problema singulariteta zbog dodatnog četvrtog člana. Kvaternionom smo također riješili problem računanja trigonometrijskih funkcija jer jednadžbe s kvaternionima su algebarskog tipa.

        Pozitivna strana prikaza orijentacije pomoću Eulerovih kutova je njihova intuitivnost jer svaki kut označava kut orijentacije oko pojedine osi. Kod kvaterniona informacija o orijentaciji je ponešto sakrivena unutar kvaterniona. Skalarni član ne označava direktno kut zakreta oko osi rotacije već njegov kosinus kuta, a vektorski dio sadrži skaliranu os rotacije sinusom kuta (vidi jednadžbu \ref{eq:kvaternion}).
    }
}

\chapter{Sustav za određivanje i upravljanje orijentacijom satelita (ADCS)}{
    U ovom poglavlju navest ćemo definiciju ADCS \engl{Attitude Determination and Control system} sustava. Navest ćemo osnovne dijelove sustava. Navest ćemo najčešće korištene senzore i aktuatore, objasniti njihove karakteristike, navest prednosti i mane. Također ćemo opisati načine i algoritme određivanja i kontrole orijentacije. Naposljetku ćemo opisati sklopovlje koje smo razvili i navesti korištene senzore i aktuatore. 

    \section{Uvod}{
        ADCS  sustav je sustav koji određuje orijentaciju satelita pomoću podataka iz senzora koji se nalaze na samom satelitu i na temelju tih podataka vrši korekciju orijentacije u željeni položaj. 

        \subsection{Određivanje i upravljanje orijentacijom}{
            Senzori koje koristi ADCS sustav za određivanje orijentacije najčešće ne daju direktno vrijednost orijentacije već se orijentacija estimira pomoću estimacijskih algoritama. Problem senzora općenito je njihova neodređenost koja onda doprinosi da je estimacija orijentacije pogrešna. Zbog toga smo razvili estimacijske algoritme koje uzimaju grešku senzora u obzir te tako smanjuju neodređenost estimirane orijentacije.

            Nakon što smo odredili orijentaciju, i ako ta orijentacija nije jednaka željenoj, slijedi postupak korekcije (kontrole). Postupak korekcije počinje računanjem razlike željene orijentacije i trenutno estimirane. Regulator orijentacije će pomoću te razlike orijentacije izračunati optimalnu korekciju koju je potrebno izvršiti kako bi došli u željenu orijentaciju. Korekciju orijentacije izvršavaju mehanički odnosno električni aktuatori. Aktuatori će stvoriti korekcijski moment koji satelit naposljetku dovodi u željenu orijentaciju.
            Opisana estimacija i kontrola orijentacije ADCS sustava događa se u stvarnom vremenu kroz cijeli radni vijek satelita i ni u jednom trenu se ne smije dogoditi da satelit posjeduje neželjenu ili nekontroliranu orijentaciju jer to može značiti kraj misije.
        }

        \subsection{Važnost ADCS sustava}{
            NASA-ino istraživanje je pokazalo da je ADCS sustav zaslužan za 23\% grešaka na navigacijsko-kontrolnim sustavima \engl{Guidance, Navigation and Control - GN\&C}, i da je većina tih anomalija uzrokovala kraj misije \cite{greskeNaAdcsPostotak}. U nastavku ćemo navest jedan primjer greške na ADCS sustavu.
    
            Lewis svemirska letjelica \cite{lewis} lansirana je 1997. godine s predviđenim radnim vijekom od 5 godina. Nakon postizanja uspješne orbite satelit je započeo s normalnim radom i nije ga više bilo potrebno aktivno kontrolirati. U normalnom radu solarni paneli su bili upereni prema Suncu kako bi prikupili maksimalnu količinu energije. Zbog greške prilikom rada, jedan od aktuatora je stvarao konstantni moment oko jedne osi koju senzori nisu mogli prepoznati. Nakon nekoliko dana, satelit se je počeo nekontrolirano rotirati i solarni paneli nisu prikupljali dovoljno energije. Naposljetku su inženjeri sa Zemlje uočili nekontroliranu orijentaciju ali je već bilo kasno jer se je baterija satelita potpuno ispraznila i komunikacija sa satelitom je bila zauvijek prekinuta \cite{greskeNaAdcsSlucajevi}.
        }
    }

    \section{Osnovni dijelovi}{
        Osnovni dijelovi ADCS sustava su računalo, senzori i aktuatori. U nastavku ćemo zasebno opisati svaki dio zasebno i navesti bitne karakteristike.
 
        \subsection{Računalo}{
            Računalo je zaslužno za prikupljanje i obradu podataka koje dolaze sa senzora, te kontrolu aktuatora. Prikupljanje podataka sa senzora vrši se preko komunikacijskih kanala (sabirnica). Obrada podataka se u suštini zasniva na računanju zahtjevnih matematičkih operacija koje na koncu daju podatak o trenutnoj orijentaciji. Na temelju razlike trenutne i željene orijentacije, računalo će izračunati kontrolni signal koji šalje aktuatorima također preko predviđenih sabirnica. 

            Neki od zahtjeva za ADCS računala su mala potrošnja, otpornost na radijaciju i kozmičke zrake, brzina izvođenja operacija, pouzdanost. U prilog ide i ako je računalo već korišteno u nekim prijašnjim misijama.
        }

        \subsection{Senzori orijentacije}{
        \label{subsection:senzori_orijentacije}
            Senzori orijentacije su elektroničke naprave iz kojih je moguće dobiti podatak o orijentaciji. Podatak o orijentaciji najčešće ne dolazi direktno iz senzora već se orijentacija estimira matematičkim algoritmima koji kao svoj rezultat daju estimiranu orijentaciju \cite{adcsKnjiga}. Za precizno određivanje orijentacije nekada je potrebno uzeti u obzir podatke iz više senzora odjednom. Takav način prikupljanja podataka zovemo engl. \emph{sensor fusion}. 

            U nastavku ćemo navesti najkorištenije senzore, njihove karakteristike i način rada.

            \subsubsection{Senzor Sunca}{
                Senzore Sunca dijelimo u dvije skupine: analogne i digitalne. Analogni senzori Sunca rade na principu solarnih ćelija, a podatak o vektoru sunca moguće je dobiti iz generirane struje ćelija. Slika \ref{fig:sensor_sunca_industrija} prikazuje industrijsku izvedbu jednog takvog senzora.

                \begin{figure}[htb]
                \centering
                \includegraphics[width=0.6\textwidth]{images/sensor_sunca_industrija.jpg}
                \caption{Prikaz industrijske izvedbe senzora Sunca tvrtke Bradford \cite{sunSensorBradford}.}
                \label{fig:sensor_sunca_industrija}
                \end{figure}

                Struja iz solarnih ćelija ovisna je o kutu $\theta$ što označava kut između normale senzora $\hat{\overrightarrow{\boldsymbol{n}}}$ i vektora nadolazeće zrake Sunca $\hat{\overrightarrow{\boldsymbol{v}}}$ (vidi sliku \ref{fig:sensor_sunca}) kao:

                \begin{equation}
                    i(\theta) = i(0)\cos\theta,
                \end{equation}

                Senzor Sunca ima konačni kut gledanja u obliku stošca. Sa samo jednim senzorom Sunca nije moguće odrediti vektor $\hat{\overrightarrow{\boldsymbol{v}}}$ već se primjenjuje paradigma \emph{sensor fusion}-a gdje koristimo dodatni senzor Sunca (vidi sliku \ref{fig:sensor_sunca_sensor_fusion}) \cite{adcsKnjiga}.

                \begin{figure}[htb]
                \centering
                \includegraphics[width=0.3\textwidth]{images/sensor_sunca.png}
                \caption{Prikaz modela senzora Sunca.}
                \label{fig:sensor_sunca}
                \end{figure}

                \begin{figure}[htb]
                \centering
                \includegraphics[width=0.55\textwidth]{images/sensor_sunca_sensor_fusion.png}
                \caption{Prikaz para senzora Sunca.}
                \label{fig:sensor_sunca_sensor_fusion}
                \end{figure}
            }

            \subsubsection{Troosni magnetometar}{
            \label{subsubsection:troosni_magnetometar}
                Troosni magnetometar mjeri vektor lokalnog magnetskog polja Zemlje u koordinatnom sustavu senzora. Magnetometri su relativno neprecizni senzori i za pouzdanija mjerenja potrebno ih je kalibrirati \cite{adcsKnjiga}.

                Kalibracija ima za cilj smanjiti utjecaj tzv. engl. \emph{Hard} i \emph{Soft} efekata. Prije nego objasnimo što su spomenuti efekti, važno je objasniti postupak kalibracije magnetometra. 

                Kalibracija magnetometra započinje prikupljanjem podataka iz senzora i to tako što ćemo magnetometar rotirati u svim smjerovima. Nakon što smo prikupili dovoljno podataka možemo prikazati izmjerene vektore magnetskog polja kao skupinu točaka gdje svaka točka prikazuje vrh vektora očitane vrijednosti. Prikaz očitanih vrijednosti kod idealnog magnetometra bit će raspoređeni po sferi radijusa duljine vektora (apsolutna vrijednost magnetskog polja) čije središte leži u centru koordinatnog sustava (vidi sliku \ref{fig:mag_ideal}).

                \begin{figure}[htb]
                \centering
                \includegraphics[width=0.7\textwidth]{images/mag_ideal.png}
                \caption{Točkasti prikaz mjerenja vektora Zemljinog magnetskog polja pomoću idealnog magnetometra. Vrijednosti mjerenja su u jedinici $\mu T$.}
                \label{fig:mag_ideal}
                \end{figure}

                Nakon što smo objasnili postupak prikupljanja podataka slijedi objašnjenje \emph{Hard} i \emph{Soft} efekata. \emph{Hard} efekt je pojava koja utječe na mjerenja magnetskog polja i manifestira se kao translacija ishodišta svih vrijednosti mjerenog vektora magnetskog polja za neki fiksni vektor (vidi sliku \ref{fig:mag_hard}). Mogli bismo slobodno reći da su mjerenja pristrana, odnosno imaju tzv. engl. \emph{bias}. Uzrok \emph{Hard} efekta su najčešće stacionarne interferencije okolnih metalnih dijelova na samom magnetometru ili pripadajućoj elektronici. 

                \begin{figure}[htb]
                \centering
                \includegraphics[width=0.7\textwidth]{images/mag_hard.png}
                \caption{Točkasti prikaz mjerenja vektora Zemljinog magnetskog polja pomoću magnetometra s tzv. \emph{Hard} efektom. Vrijednosti mjerenja su u jedinici $\mu T$.}
                \label{fig:mag_hard}
                \end{figure}

                \emph{Soft} efekti nastaju zbog objekata u blizini magnetometra koji onda rade distorziju lokalnog magnetskog polja. \emph{Soft} efekt kolokvijalno rečeno razvlači i kosi sferu mjerenja koja zatim oblikom postaje nakošeni elipsoid (vidi sliku \ref{fig:mag_hard_soft}).

                \begin{figure}[htb]
                \centering
                \includegraphics[width=0.7\textwidth]{images/mag_hard_soft.png}
                \caption{Točkasti prikaz mjerenja vektora Zemljinog magnetskog polja pomoću magnetometra s tzv. \emph{Hard} i \emph{Soft} efektima. Vrijednosti mjerenja su u jedinici $\mu T$.}
                \label{fig:mag_hard_soft}
                \end{figure}

                Modeliranje spomenutih efekata opisujemo u sljedećoj jednadžbi:

                % TODO: dodaj boldanje
                \begin{equation}
                    (x - b)R(x - b)^T = \beta^2,
                \end{equation}

                gdje je $R$ matrica tipa $3\times3$ i određuje oblik elipsoida (npr. za sferu jedinična matrica, a za elipsoid pozitivno definitna). $b$ je $1\times3$ vektor koji definira središte elipsoida, a $\beta$ je skalar koji opisuje duljinu vektora magnetskog polja. 

                Kalibracija magnetometra vrši se pomoću sljedeće relacije:

                \begin{equation}
                    m = (x - b)A,
                \end{equation}

                gdje A matrica tipa $3\times3$ transformira oblik elipsoida u pravilnu sferu (eliminira utjecaj \emph{Soft} efekta), a već definirani vektor $b$ translatira elipsoid u ishodište koordinatnog sustava (eliminira utjecaj \emph{Hard} efekta).

                Problem pronalaska $A$ i $b$ parametra rješavamo pomoću \texttt{magcal} MATLAB funkcije \cite{magcal}, a kalibraciju vršimo u programskom kodu netom nakon primitka nekalibrirane vrijednosti s magnetometra \cite{kalibracijaMagKod}.

                Dodatna objašnjenja u vezi kalibracije magnetometra možete pronaći ovdje \cite{kalibracijaMatlabStranica}.

                Zbog vanjskog utjecaja na vrijednosti mjerenja magnetometra, magnetometar se može fizički udaljiti od satelita pomoću konstrukcije nazvane engl. \emph{boom} (vidi sliku \ref{fig:boom}) te tako smanjiti pogreške mjerenja \cite{adcsKnjiga}. 

                \begin{figure}[htb]
                \centering
                \includegraphics[width=0.5\textwidth]{images/boom.jpg}
                \caption{Prikaz tzv. \emph{boom}-a na čijem vrhu se nalazi magnetometar. Izvor Ørsted misija \cite{boomCite}.}
                \label{fig:boom}
                \end{figure}
            }

            \subsubsection{Pratioci zvijezda \engl{Star Trackers}}{
                Pratioci zvijezda odnosno \emph{Star Track}-eri pružaju najpreciznija mjerenja orijentacije iz dva razloga: prvi je što su svjetlosni izvori zvijezda točkaste prirode i tako pružaju precizna mjerenja. Drugi razlog je taj što se zvijezde, zbog svoje udaljenosti od senzora, doimaju stacionarne neovisno o položaju satelita unutar sunčevog sustava (čak i šire).

                \begin{figure}[htb]
                \centering
                \includegraphics[width=0.5\textwidth]{images/star_tracker.png}
                \caption{Prikaz tzv. \emph{boom}-a na čijem vrhu se nalazi magnetometar. Izvor Redwire \cite{starTrackerCite}.}
                \label{fig:star_tracker}
                \end{figure}

                Pratioci zvijezda mogu pratiti vektor smjera jedne zvijezde, i u ovisnosti o položaju senzora na satelitu, orijentacijskom matricom moguće dobiti podatak smjera zvijezde u referentnom sustavu satelita. Pomoću dodatnih senzora moguće je jednoznačno odrediti orijentaciju satelita. 

                U praksi pratioci zvijezda prate više zvijezda odjednom i pomoću zvjezdanih kataloga \cite{starCatalogs} određuje direktno podatak o orijentaciji. Jednom kada je senzor odredio orijentaciju, svako sljedeće mjerenje zasniva se na praćenju istih zvijezda i metodom usrednjavanja moguće je postignuti veću preciznost. 

                Zbog navedenih razloga, pratioci zvijezda su najprecizniji senzori orijentacije od svih do sad navedenih senzora. S druge strane su i najkompleksniji, najskuplji i najnepouzdaniji senzori \cite{adcsKnjiga}. 
            }
        }

        \subsection{Senzori kutne brzine}{
        \label{subsection:senzori_kutne_brzine}
            Senzori kutne brzine mjere kutnu brzinu satelita. Kao što ćemo vidjeti u nastavku rada, većina algoritama za procjenu orijentacije koriste kutnu brzinu kao dodatnu informaciju. Kutna brzina može se izračunati kao vremenska derivacija orijentacije, ali takav izračun unosi neodređenost u obliku šuma. Senzori kutne brzine nemaju izražen šum, ali zato imaju problem s tzv. \emph{bias}-om - konstantnoj srednjoj vrijednosti koja unosi grešku u račun. 

            Osim niskog šuma, druga prednost senzora kutne brzine je što ne zahtijevaju eksterni izvor informacije. Primjerice, senzor Sunca uvijek mora imati prisutno Sunce ne bi li izračunali orijentaciju satelita. U tim situacijama kada ne možemo dobiti direktno podatak o orijentaciji satelita, integracijom kutne brzine moguće je odrediti orijentaciju. Problem takvog načina određivanja orijentacije je konstantni \emph{bias} koji unosi grešku u integraciju. Zbog toga se takav način određivanja orijentacije ograničava na kratke vremenske periode u kojima je greška zbog integracije konstantnog člana zanemariva.

            Senzora kutne brzine ima više vrsta, a tradicionalno su najpopularniji mehanički žiroskopi. Mane mehaničkih žiroskopa su mehanički pokretni dijelovi koji stvaraju vibracije, s vremenom im raste nepouzdanost i relativno velika potrošnja energije. Zbog toga, nedavno su u primjenu ušli laserski žiroskopi koji ne posjeduju pokretne dijelove \cite{adcsKnjiga}. 

            Osim mehaničkih i laserskih žiroskopa, postoje i tzv. \emph{MEMS} žiroskopi koji pomoću Coriolisovog efekta mjere kutnu brzinu. Takvi senzori najčešće se ugrađuju u integrirane sklopove i u svom radu koriste nanostrukture koje omogućavaju mjerenje Coriolisovog efekta \cite{memsGyro}.
        }

        \subsection{Aktuatori}{
            Aktuatori su uređaji koji na temelju kontrolnog signala vrše korekciju orijentacije. Dijelimo ih u dvije skupine: one koji mijenjaju kutnu količinu gibanja satelita i one koji ne mijenjaju. Potisnici i magnetorkeri spadaju u prvu skupinu, a zamašnjaci u drugu skupinu \cite{adcsKnjiga}. Slijedi opis svakog spomenutog aktuatora. 

            \subsubsection{Potisnici \engl{Thrusters}}{
                Potisnici su aktuatori koji rade na principu izbacivanja mase ne bi li stvorili potisnu silu. Ako potisni vektor ne prolazi kroz centar mase satelita, u trenutku izbacivanja mase doći će do generiranja momenta. Kao što se može vidjeti na slici \ref{fig:thruster_img}, potisnik koji generira silu $F$ na udaljenosti $r$ od centra mase, uzrokuje moment $T$ definiran kao $T=Fr$.

                \begin{figure}[htb]
                \centering
                \includegraphics[width=0.5\textwidth]{images/thruster_img.png}
                \caption{Princip rada potisnika.}
                \label{fig:thruster_img}
                \end{figure}

                Zbog toga što potisnik nije u stanju generiranja sile $F$ u oba smjera (naravni i suprotni smjer), za stvaranje pozitivnog i negativnog momenta potrebno je imati dva potisnika koja su međusobno suprotno orijentirana. Sukladno tomu, za troosno upravljanje orijentacijom potrebno je imati 6 potisnika \cite{adcsKnjiga}.
            }

            \subsubsection{Magnetorkeri \engl{magnetorquers, magnetic torquers}}{
                Magnetorkeri su aktuatori koji generiraju moment pomoću magnetske sprege lokalno generiranog magnetskog polja i magnetskog polja Zemlje. Magnetorker je u suštini zavojnica. Kao što nalaže Amperov zakon, prolaskom struje kroz namotaje zavojnice dolazi do stvaranja lokalnog magnetskog polja (magnetski dipol). Međusobnom interakcijom generiranog magnetskog dipola magnetorkera $\overrightarrow{\boldsymbol{m}}$ i magnetskog dipola Zemlje $\overrightarrow{\boldsymbol{b}}$ stvara se moment $\overrightarrow{\boldsymbol{T}}$ koji je jednak:

                \begin{equation}
                    \label{eq:magnetorker_eq}
                    \overrightarrow{\boldsymbol{T}} = \overrightarrow{\boldsymbol{m}} \times \overrightarrow{\boldsymbol{b}}
                \end{equation}

                Moguće je pokazati kako je vektorski produkt u jednadžbi \ref{eq:magnetorker_eq} (isto vektor) uvijek okomit na magnetski dipol Zemlje $\overrightarrow{\boldsymbol{b}}$. Zbog toga je nemoguće generirati moment kada je položaj dipola magnetorkera usmjeren kao i dipol magnetskog polja Zemlje.

                Magnetorker omogućuje upravljanje orijentacijom samo u jedno osi, što znači da za troosno upravljanje orijentacijom tipično imamo 3 međusobno ortogonalna magnetorkera za svaku os zasebno (vidi sliku \ref{fig:magnetorquer_img}).

                \begin{figure}[htb]
                \centering
                \includegraphics[width=0.7\textwidth]{images/magnetorquer_img.jpg}
                \caption{Magnetorker tvrtke NanoAvionics \cite{magnetorquer_cite}.}
                \label{fig:magnetorquer_img}
                \end{figure}

                Zbog relativno slabe jakosti Zemljinog magnetskog polja, teško je generirati momente s dovoljnom jačinom jakosti specificirane zahtjevima, pogotovo za satelite koji posjeduju velike momente tromosti. Zbog toga najčešće se kontrola orijentacije vrši pomoću zamašnjaka (vidi kasnije poglavlje) u sprezi sa magnetorkerima iz dva razloga. Prvi razlog je već spomenuta nedovoljna jakost generiranog momenta od strane magnetorkera, i drugi razlog, problem zasićenja zamašnjaka.

                Valja skrenuti pažnju i na problem utjecaja magnetorkera na mjerenja senzora magnetskog polja. Ako prilikom korekcije orijentacije pomoću magnetorkera mjerimo magnetsko polje, doći će do magnetske interferencije magnetorkera i senzora magnetskog polja (vidi poglavlje \ref{subsubsection:troosni_magnetometar}). Samim time direktno unosimo grešku u estimaciji orijentacije. Rješenje problemu leži u tome da za vrijeme mjerenja magnetskog polja magnetorker ne vrši korekciju orijentacije \cite{adcsKnjiga}.

                U sklopu projekta razvili smo metodu optimalne parametrizacije magnetorkera koja za dani magnetski moment vrši parametrizaciju magnetorkera tako da posjeduje minimalnu masu uz minimalni potrošak snage \cite{magnetorker_ieee}.
            }

            \subsubsection{Zamašnjaci \engl{Momentum wheels}}{
                Zamašnjaci su mehanički uređaji koji rade na principu očuvanja kutne količine gibanja. Zamašnjak se sastoji od rotirajuće mase i elektromotora. Naime, kada elektromotorom vršimo moment u jednom smjeru, radi očuvanja kutne količine gibanja, zamašnjak će generirati jednak moment po iznosu ali u suprotnom smjeru ne bi li ukupna kutna količina gibanja satelita ostala jednaka (vidi sliku \ref{fig:zamasnjak_fig}). U nominalnom stanju, zamašnjak ne posjeduje kutnu brzinu. Zamašnjaci kao aktuatori predstavljaju najprecizniju kontrolu orijentacije \cite{adcsKnjiga}.

                \begin{figure}[htb]
                \centering
                \includegraphics[width=0.7\textwidth]{images/zamasnjak.png}
                \caption{Princip rada zamašnjaka.}
                \label{fig:zamasnjak_fig}
                \end{figure}

                Problem zamašnjaka manifestira se s vremenom. Da bi objasnili problem pretpostavimo da satelitu želimo održavati jednaku orijentaciju bez obzira na vanjske utjecaje (ne ulazeći u uzroke). Svaki vanjski utjecaj (smetnja) na satelit uzrokuje generiranim momentom. Kako bismo satelitu održali jednaku orijentaciju, zamašnjakom upravljamo tako da ćemo generirati jednaki moment ali u suprotnu stranu (zamašnjak će se zarotirati u istu stranu kao i vanjski moment). Jednom kada smo pomoću zamašnjaka uveli jednaku kutnu količinu gibanja kao što ju je uvela vanjska smetnja, zamašnjaku moramo konstantno održavati kutnu brzinu. Bilo kakva promjena kutne brzine uzrokuje promjenu kutne količine gibanja odnosno promjenu orijentacije. Ako se s vremenom smetnje akumuliraju, dolazi do pojave tzv. saturacije zamašnjaka gdje se zamašnjak nastoji okretati sve većom i većom brzinom sve dok ne dosegne maksimalnu moguću kutnu brzinu. Drugi suptilni problem je taj što rotirajućem zamašnjaku stalno moramo održavati kutnu brzinu što donosi negativnu energetsku računicu. Desaturaciju zamašnjaka vršimo pomoću drugih aktuatora (npr. magnetorkera).

                Drugi problem zamašnjaka leži u činjenici da su zamašnjaci mehanički uređaji s rotirajućim dijelovima te se očekuje da će s vremenom prestati raditi. Iz tog razloga umjesto 3 zamašnjaka za troosno upravljanje orijentacijom, uzimamo 4 zamašnjaka karakteristično položena (vidi sliku \ref{fig:cetiri_zamasnjaka}). Tako je moguće imati troosno upravljanje orijentacijom s jednim zamašnjakom koji ne radi \cite{cetiriZamasnjaka}.

                \begin{figure}[htb]
                \centering
                \includegraphics[width=0.5\textwidth]{images/cetiri_zamasnjaka.jpg}
                \caption{Skup od 4 zamašnjaka tvrtke NanoAvionics gdje je jedan zamašnjak redundantan \cite{cetiriZamasnjakaTvrtka}.}
                \label{fig:cetiri_zamasnjaka}
                \end{figure}
            }
        }
    }

    \section{Algoritmi za određivanje i upravljanje orijentacijom}{
        U ovom dijelu navest ćemo neke algoritme i metode za određivanje i upravljanje orijentacijom satelita. 
 
        \subsection{Određivanje orijentacije}{
        \label{subsection:odredjivanje_orijentacije}
            Određivanje orijentacije je postupak u kojem iz odabranih orijentacijskih senzora (vidi poglavlje \ref{subsection:senzori_orijentacije}) prikupljamo podatke koje potom obrađujemo unutar posebno razvijenih algoritama koji kao rezultat računanja na izlazu daju podatak o orijentaciji. Orijentacija satelita nije jednoznačno određena već u njoj postoje neodređenosti (greške). Takve neodređenosti najviše proizlaze iz neodređenosti senzora iz kojih prikupljamo podatke. Druge, manje utjecajnije neodređenosti orijentacije proizlaze iz konačne preciznosti aritmetičkih operacija procesora na kojima se izvršavaju takvi algoritmi. Zbog svega toga, vrlo često u radu pišemo da orijentaciju estimiramo jer je podatak o orijentaciji (više ili manje) gruba procjena. Kako bi smanjili neodređenosti procjene orijentacije, najčešće uzimamo podatke iz više neovisnih senzora i vršimo postupak \emph{Sensor fusion}-a. \emph{Sensor fusion} će smanjiti neodređenosti pojedinih senzora te u konačnici dati bolji rezultat o procjeni orijentacije nego što bismo dobili kad bismo koristili podatak iz samo jednog senzora.

            Algoritmi za estimaciju orijentacije vrlo često budu razvijeni tako da optimalno procjenjuju orijentaciju s obzirom na neki uvjet koji je unaprijed zadan. Možda najpoznatiji problem optimalne estimacije orijentacije satelita naziva se tzv. \emph{Wahba's Problem} iz 1965. godine \cite{wahbas_problem}. Nekoliko algoritama je razvijeno za potrebe rješavanja \emph{Wahba's} problema. Neke od njih navest ćemo u nastavku poglavlja.

            Pri estimaciji orijentacije važno je shvatiti nekoliko stvari. Algoritmi koje ćemo spomenuti, za svoj rad koriste podatke iz 2 različita troosna senzora položena u referentni sustav tijela (satelita). Za svaki od ta dva senzora unaprijed moraju biti poznate njihove tzv. referentne vrijednosti. Referentne vrijednosti senzora označavaju vektore koje smo unaprijed dobili ili izračunali. Razlika trenutno dobivene vrijednosti iz senzora i njegove referentne vrijednosti kolokvijalno nam govori za koliko smo zaokrenuli tijelo. Za lakše shvaćanje pojma referentnog vektora navest ćemo primjer referentnog vektora akcelerometra. Kao dogovor uzet ćemo primjer gdje je referentni vektor akcelerometra položen u referentni sustav kojemu Z komponenta gleda prema središtu Zemlje (X i Y komponenta za ovaj primjer nisu važne). U tom slučaju vrijedi kako će očitana vrijednost s akcelerometra biti jednaka $\left[0 \; 0 \; -9.81\right] m/s^2$. Recimo da u jednom trenu promijenimo orijentaciju senzora tako da mu Z komponenta gleda u suprotnom smjeru. U tom ćemo slučaju s akcelerometra pročitati vrijednost mjerenja iznosa $\left[0 \; 0 \; +9.81\right] m/s^2$ iz čega možemo zaključiti kako se je orijentacija akcelerometra promijenila.

            Algoritme za estimaciju orijentacije dijelimo u dvije skupine: bezmemorijske i memorijske algoritme \cite{uvod_u_svemirske}. Bezmemorijski algoritmi ne uzimaju u obzir prošlo stanje orijentacije već podatak o orijentaciji estimiraju samo iz trenutnih vrijednosti senzora (naravno i njihovih referentnih vrijednosti). Memorijski algoritmi će na temelju prošlog stanja orijentacije i trenutnih vrijednosti senzora estimirati orijentaciju. Memorijski algoritmi su superiorniji jer rade na ideji rekurzije. Svaka iznova estimirana orijentacija biva sve točnija baš zato što senzor uzima u obzir i prošlu orijentaciju, a po logici očuvanja stanja sustava, prošlo stanje je ili isto ili vrlo slično trenutnom stanju. 

            Popularni bezmemorijski algoritam ima naziv QUEST, a memorijski REQUEST, Optimal-REQUEST, Komplementarni filtar i dr. O njima nešto više u nastavku. 

            \subsubsection{Wahba problem}{
                Da bismo objasnili estimacijske algoritme, prvo je potrebno da definiramo Wahba problem. Ako su nam dani dva skupa vektora sa $n\ge2$ elemenata $\left\{ \boldsymbol{r}_1, \boldsymbol{r}_2, \cdots , \boldsymbol{r}_n \right\}$ i $\left\{ \boldsymbol{b}_1, \boldsymbol{b}_2, \cdots , \boldsymbol{b}_n \right\}$, Wahba problem je problem pronalaženja matrice $A$ koja će transformirati prvi set vektora u drugi na način koji najbolje minimizira srednju kvadratnu pogrešku relacije:

                \begin{equation}
                    \sum_{i=1}^{n}||\boldsymbol{b}_i - A\boldsymbol{r}_i||^2,
                \end{equation}

                gdje za matricu $A$ vrijedi $A^T A = \boldsymbol{I}_3$ i $\det(A)=1.$
            }

            \subsubsection{REQUEST algoritam}{
                REQUEST algoritam je algoritam koji rješava Wahba problem pronalaska orijentacijske matrice $A$. REQUEST algoritam je preteča QUEST algoritma koji je uzimao u obzir samo $n$ trenutno izmjerenih vektora i na temelju njih izračunao orijentacijsku matricu \cite{quest_algo}. 

                REQUEST algoritam ide korak dalje i uvodi rekurziju. Na temelju već dobivene orijentacijske matrice u prijašnjem $k$-tom koraku i novo izmjerenih $\boldsymbol{b}_i$ i $\boldsymbol{r}_i$ vektora senzora orijentacije, REQUEST algoritam računa orijentacijsku matricu u $k+1$ koraku \cite{request_algo}.

                Osim izmjerenih $\boldsymbol{b}_i$ i $\boldsymbol{r}_i$ vektora, REQUEST algoritam uzima u obzir i trenutnu kutnu brzinu tijela pomoću koje točnije estimira orijentacijsku matricu. 

                Jedan problem REQUEST algoritma je taj što zahtijeva par dodatnih (fiksnih) koeficijenata koji moraju biti empirički izabrani. Ti koeficijenti su zapravo težinski faktori koji govore kojim mjerenjima pridodajemo više važnosti: senzorima orijentacije ili mjerenjima žiroskopa. Zbog tih parametara REQUEST algoritam je suboptimalan \cite{opt_req_algo}. Problem suboptimalnosti rješava Optimal-REQUEST algoritam kojeg ćemo navesti u nastavku.

                Valja još spomenuti kako REQUEST algoritam nije općeniti algoritam koji će na izlazu dati estimaciju spomenute matrice $A$, već je to algoritam koji je specifično razvijen da na svom izlazu daje podatak o orijentaciji u obliku kvaterniona.
            }

            \subsubsection{Optimal-REQUEST algoritam}{
                Optimal-REQUEST algoritam je preteča REQUEST algoritma. Kao što smo već napomenuli ranije, REQUEST algoritam sadrži empirički izabrane koeficijente zbog kojih je takav algoritam suboptimalan. Optimal-REQUEST algoritam će na optimalan način s vremenom, s obzirom na propagaciju greške na senzorima orijentacije i žiroskopu, sam estimirati spomenute koeficijente. Metoda estimacije koeficijenata zasniva se na Kalmanovom filtru \cite{opt_req_algo}.

                Nećemo ulaziti u detalje implementacije već ćemo samo navesti kako smo u sklopu ovog projekta razvili Optimal-REQUEST algoritam u MATLAB kodu, te smo ga naposljetku pretvorili u C kod. Cijeli programski kod dostupan je ovdje \cite{opt_req_kod} uz popratnu dokumentaciju ovdje \cite{opt_req_dokumen}. 
            }

            \subsubsection{Gradijentni spust \engl{Gradient descent}}{
                Ako posjedujemo dva para vektora mjerenja $(\boldsymbol{v}_1^i, \; \boldsymbol{v}_1^b)$ i $(\boldsymbol{v}_2^i, \; \boldsymbol{v}_2^b)$ gdje $\boldsymbol{v}_i^b$ označava vektor definiran u referentnom sustavu tijela $\mathcal{F}_b$, a $\boldsymbol{v}_i^i$ označava vektor definiran u inercijskom referentnom sustavu $\mathcal{F}_i$, moguće je definirati tzv. rotacijsku matricu $\mathcal{R}_i^b$ (vidi jednadžbu \ref{eq:rot_mat_quat}) koja transformira vektor $\boldsymbol{v}^i $ u vektor $\boldsymbol{v}^b$. 

                Cilj estimacijskog algoritma je izračunati orijentacijsku matricu tako da što više smanjimo pogrešku estimacije. Pogrešku estimacije $\boldsymbol{e_i} = \left[e_{x,i} \; e_{y,i} \; e_{z,i}\right]^T$ za svaki par mjerenih vektora definiramo kao:

                \begin{equation}
                \begin{array}{rcl}
                    \boldsymbol{e}_1 & = & \mathcal{R}_i^b \boldsymbol{v}_1^i - \boldsymbol{v}_1^b, \\
                    \boldsymbol{e}_2 & = & \mathcal{R}_i^b \boldsymbol{v}_2^i - \boldsymbol{v}_2^b. 
                \end{array}
                \end{equation} 

                Nadalje, definirat ćemo funkciju cilja. Funkcija cilja u gradijentnom spustu je parametar kojeg želimo maksimalno smanjiti. Funkciju cilja definiramo kao:

                \begin{equation}
                    J(\boldsymbol{q}_i^b) = \boldsymbol{e}_1^T \boldsymbol{e}_1 + \boldsymbol{e}_2^T \boldsymbol{e}_2
                \end{equation}

                Konačno, algoritam gradijentnog spusta definiramo kao:

                \begin{equation}
                \label{eq:grad_des}
                    \boldsymbol{q}_{n+1} = \boldsymbol{q}_n - \alpha\nabla J(\boldsymbol{q}_n),
                \end{equation}

                gdje je gradijent funkcije cilja $\nabla J(\boldsymbol{q}_n)$ definiran kao:

                \begin{equation}
                    \nabla J(\boldsymbol{q}_n) = 
                    \begin{bmatrix}
                        2 \boldsymbol{e}_1^T \boldsymbol{M}_1 \boldsymbol{v}_1^i(\boldsymbol{q}_n) + 2 \boldsymbol{e}_2^T \boldsymbol{M}_1 \boldsymbol{v}_2^i(\boldsymbol{q}_n) \\
                        2 \boldsymbol{e}_1^T \boldsymbol{M}_2 \boldsymbol{v}_1^i(\boldsymbol{q}_n) + 2 \boldsymbol{e}_2^T \boldsymbol{M}_2 \boldsymbol{v}_2^i(\boldsymbol{q}_n) \\
                        2 \boldsymbol{e}_1^T \boldsymbol{M}_3 \boldsymbol{v}_1^i(\boldsymbol{q}_n) + 2 \boldsymbol{e}_2^T \boldsymbol{M}_3 \boldsymbol{v}_2^i(\boldsymbol{q}_n) \\
                        2 \boldsymbol{e}_1^T \boldsymbol{M}_4 \boldsymbol{v}_1^i(\boldsymbol{q}_n) + 2 \boldsymbol{e}_2^T \boldsymbol{M}_4 \boldsymbol{v}_2^i(\boldsymbol{q}_n)
                    \end{bmatrix}
                    ,
                \end{equation}

                i gdje su sve parcijalne derivacije rotacijske matrice $\mathcal{R}_i^b$ definirane kao:

                \begin{equation}
                \begin{array}{rcl}
                    \boldsymbol{M}_1 = \frac{\partial \mathcal{R}_i^b}{\partial q_1} = 2
                    \begin{bmatrix}
                        q_1 & q_4 & -q_3 \\
                        -q_4 & q_1 & q_2 \\
                        q_3 & -q_2 & q_1 
                    \end{bmatrix},
                    & 
                    \boldsymbol{M}_2 = \frac{\partial \mathcal{R}_i^b}{\partial q_2} = 2
                    \begin{bmatrix}
                        q_2 & q_3 & q_4 \\
                        q_3 & -q_2 & q_1 \\
                        q_4 & -q_1 & -q_2 
                    \end{bmatrix},
                    \\
                    & & \\
                    \boldsymbol{M}_3 = \frac{\partial \mathcal{R}_i^b}{\partial q_3} = 2
                    \begin{bmatrix}
                        -q_3 & q_2 & -q_1 \\
                        q_2 & q_3 & q_4\\
                        q_1 & q_4 & -q_3 
                    \end{bmatrix},
                    & 
                    \boldsymbol{M}_4 = \frac{\partial \mathcal{R}_i^b}{\partial q_4} = 2
                    \begin{bmatrix}
                        -q_4 & q_1 & q_2 \\
                        -q_1 & -q_4 & q_3 \\
                        q_2 & q_3 & q_4 
                    \end{bmatrix}.
                \end{array}
                \end{equation}

                Procjena orijentacijskog kvaterniona započinje odabirom inicijalnog kvaterniona $\boldsymbol{q}_0$. Za inicijalni kvaternion najčešće se uzima proizvoljna vrijednost (ako drugačije nije moguće) npr. $\boldsymbol{q}_0 = \left[1, \; 0, \; 0, \; 0\right]^T$. Svaki sljedeći korak uključuje potraživanje mjerenih vektora iz orijentacijskih senzora i računanje procjene orijentacije jednadžbom \ref{eq:grad_des}.

                Problem opisane procijene orijentacijskog kvaterniona metodom gradijentnog spusta je što takva procjena sadrži visokofrekventne komponente šuma. Takav šum moguće je filtrirati niskopropusnim filtrom. Međutim, postoji bolji način filtriranja. Naime, kao što je bilo rečeno u poglavlju \ref{subsection:senzori_kutne_brzine}, orijentaciju je moguće izračunati iz mjerenog vektora kutne brzine (vidi jednadžbu \ref{eq:rot_qib_elem}) relacijom \cite{uvod_u_svemirske}:

                \begin{equation}
                    \boldsymbol{q}_k = \boldsymbol{q}_{k-1} + \Delta T \dot{\boldsymbol{q}}_k = \boldsymbol{q}_{k-1} + \frac{\Delta T}{2} \boldsymbol{S}(\boldsymbol{\omega}_{bG}) \boldsymbol{q}_{k-1},
                \end{equation}

                gdje je $\Delta T$ proteklo vrijeme od prošlog računanja.

                Međutim, problem takvog načina određivanja orijentacije je akumulirana greška. Takvu akumuliranu grešku moguće je filtrirati visokopropusnim filtrom. 

                Nadalje, pokušat ćemo spojiti obje metode procijene orijentacije u jednu metodu ne bi li tako u konačnici dobili procjenu s manjom greškom. U tu svrhu uvest ćemo tzv. komplementarni filtar.
            }

            \subsubsection{Komplementarni filtar}{
                Komplementarni filtar zasniva se na jednostavnoj ideji gdje izlaz filtra ovisi o dvoma ulaznim vrijednostima i to tako da filtru težinskim faktorom $K \in \left[ 0, \; 1\right]$ odredimo hoće li izlaz filtra više ovisiti o prvom, odnosno drugom ulaznom parametru. 

                Cijela logika računanja orijentacijskog kvaterniona može se opisati slikom \ref{fig:komp_filt_fig}. Obratimo pažnju na to kako, s obzirom na težinski faktor $K$, izlaz filtra ovisi o jednom ili o drugom ulaznom parametru. Ako vrijedi da je $K = 1$, izlaz komplementarnog filtra, odnosno procijenjena orijentacija, ovisi isključivo o algoritmu određivanja orijentacije pomoću vektora kutne brzine. Vrijedi i obrnuto, ako je $K = 0$, izlaz isključivo ovisi o algoritmu procijene orijentacije pomoću gradijentnog spusta \cite{grad_desc}. Parametar $K$ moguće je odrediti eksperimentalnim putem ili simulacijom. 

                \begin{figure}[htb]
                \centering
                \includegraphics[width=1.0\textwidth]{images/komp_filt_fig.png}
                \caption{Prikaz komplementarnog filtra za određivanje orijentacije satelita pomoću gradijentnog spusta i vektora kutne brzine \cite{uvod_u_svemirske}.}
                \label{fig:komp_filt_fig}
                \end{figure}
            }
        }

        \subsection{Upravljanje orijentacijom/kutnom brzinom}{
            \subsubsection{Pojednostavljeni model satelita}{
                Zbog lakšeg razvijanja regulacijskih petlji odlučili smo se za upravljanje orijentacijom, odnosno kutnom brzinom samo oko jedne osi. 

                Ako vrijedi da su kutne brzine $TODO$ i $TODO$, iz jednadžbe \ref{eq:} vidimo kako vrijedi:

                \begin{equation}
                \end{equation}

                Ta jednadžba kutne brzine satelita podrazumijeva kako upravljački moment zamašnjaka $TODO$ može biti beskonačne vrijednosti što nije slučaj zbog toga što zamašnjak može generirati konačnu kutnu količinu gibanja. Zbog toga uvodimo drugi pojednostavljeni model satelita koji se zasniva na modelu prvog reda:

                \begin{equation}
                    TODO,
                \end{equation}

                gdje parametar $TODO$ označava maksimalnu kutnu brzinu s obzirom na upravljački signal PWM-a, a parametar $TODO$ možemo usporediti s tromosti sustava.
                
                Drugi pojednostavljeni model sustava opisuje nam orijentaciju satelita oko jedne osi, odnosno kut i definiramo je kao:

                \begin{equation}
                    TODO.
                \end{equation}

                TODO: dodat graf eksponencijalne funkcije
            }

            \subsubsection{Korišteni aktuatori}{
                U našem radu za kontrolu orijentacije upotrijebili smo zamašnjak. Zamašnjak je pogonjen elektromotorom. Elektromotor se upravlja pomoću upravljivog H-mosta koji omogućuje promjenu smjera. Brzinom elektromotora upravlja se pomoću pulsno širinske modulacije \engl{Pulse Width Modulation - PWM}. Rotirajuća masa izrađena je na 3D printeru.

                Razlog zašto smo odabrali zamašnjak za upravljanje orijentacijom je u tome što je to najlakši aktuator za izvest. Magnetorker bi zahtijevao izradu posebnog kaveza koji zamjenjuje Zemljino magnetsko polje što dodatno komplicira testni sustav.

                PWM modulacija generira se u sklopovlju mikroupravljača.
            }

            \subsubsection{Korišteni algoritmi}{
                Upravljanje orijentacijom satelita prvo započinje određivanjem orijentacije. Jednom kada je orijentacija određena, upravljački PWM signal upravlja elektromotorom.

                Upravljački signal $u(t)$ ovisi o razlici trenutne i željene orijentacije $e(t)$. Razliku trenutne i željene orijentacije dovodimo na ulaz PID regulatora koji zatim na izlazu daje upravljački signal $u(t)$. Upravljački signal PID regulatora je u intervalu $u(t) \in \left[ 0, 100 \right]$ što nam direktno daje tzv. \emph{Duty cycle} (DC) PWM-a. DC računamo kao \cite{dc_pwm}:

                \begin{equation}
                    DC = \frac{T_{on}}{T_{on} + T_{off}} \times 100 \left[ \% \right],
                \end{equation}

                gdje je parametar $T_{on}$ jednak vremenu visokog stanja PWM signala, a parametar $T_{off}$ jednak vremenu niskog stanja PWM signala. Zbroj vremena visokog i niskog stanja PWM daje nam period PWM signala $T_{\text{PWM}} = T_{on} + T_{off}$. Ako vrijedi da je $T_{on}=T_{\text{PWM}}$ onda znači da je srednja vrijednost PWM signala jednaka naponu napajanja PWM-a. Vrijedi i obrnuto, ako vrijedi da je $T_{off}=T_{\text{PWM}}$ onda srednja vrijednost PWM signala iznosi 0.

                U našem sustavu implementirane su dvije regulacijske petlje: prva koja upravlja kutnom brzinom satelita i druga koja upravlja orijentacijom. Radi jednostavnosti obje regulacijske petlje reguliraju orijentaciju/kutnu brzinu samo oko jedne osi.

                PID regulator kojeg koristimo oblika je:

                \begin{equation}
                    u(t) = K_p e(t) + K_i \int_{}^{}{e(t)dt} + K_d \frac{de(t)}{dt},
                \end{equation}

                gdje su $K_p$, $K_i$ i $K_d$ koeficijenti PID regulatora.

                Za implementaciju PID regulatora u računalu pogodan je diskretizirani oblik. Diskretizirani oblik PID regulatora \cite{diskr_pid} definiramo kao:

                \begin{equation}
                    u(k) = K_P e(k) + K_I \sum_{}^{}{e(k)} + K_D (e(k) - e(k-1)),
                \end{equation}

                gdje su koeficijenti diskretiziranog regulatora jednaki:

                \begin{equation}
                    \begin{array}{rcl}
                        K_P & = & K_p,\\
                        K_I & = & \frac{K_p T}{T_i}, \\
                        K_D & = & \frac{K_p T_d}{T},
                    \end{array}
                \end{equation}

                i gdje je parametar $T$ jednak vremenskom periodu pozivanja regulacijske petlje.

                Odabir PID koeficijenata za obje regulacijske petlje opisan je u poglavlju \ref{}.
            }
        }
    }

    \section{Razvijeni sustav}{
        U ovom poglavlju bit će opisana razvijena tiskana pločica i sklopovlje. Također, navest ćemo izabrane orijentacijske senzore i aktuatore.

        Cijeli razvijeni sustav unutar polusferične kugle položen na zračni ležaj prikazan je na slici \ref{fig:slika_sustava}.

        \begin{figure}[htb]
        \centering
        \includegraphics[width=0.5\textwidth]{images/slika_sustava.png}
        \caption{Prikaz ADCS sustava sa pripadajućim zračnim ležajem.}
        \label{fig:slika_sustava}
        \end{figure}

        \subsection{Tiskana pločica i sklopovlje}{
            Tiskana pločica sa sklopovljem prikazana na slici \ref{fig:plocica_3d} i sastoji se od nekoliko sastavnih dijelova:

            \begin{itemize}
            \item Baterija,
            \item Napajanje,
            \item Drajver za zamašnjak,
            \item Elektromotor,
            \item Bluetooth modul,
            \item Inkrementalni enkoder,
            \item IMU modul,
            \item Nucleo pločica.
            \end{itemize}

            \begin{figure}[htb]
            \centering
            \includegraphics[width=1.0\textwidth]{images/plocica_3d.jpg}
            \caption{3D prikaz tiskane pločice ADCS sustava.}
            \label{fig:plocica_3d}
            \end{figure}

            Shema tiskane pločice nalazi se na slici \ref{fig:plocica_shema}. Napon sa baterija dolazi do napajačkog sklopovlja (oznaka \texttt{Buck converter module}) koji smanjuju napon na razine pogodne za napajanje sklopovlja. Drajver za zamašnjak (\texttt{Motor driver module}) oznake Pololu DRV8838 \cite{pololu} je mali H-most modul za upravljanje elektromotorom zamašnjaka. Elektromotor je male snage i nalazi se u samoj sredini pločice (\texttt{Motor}). Na pločici se također nalazi i Bluetooth modul oznake TODO (\texttt{Bluetooth module}). Njegov položaj je na samom rubu pločice radi bolje transmisije antene koja se nalazi na istom modulu u obliku PCB linije. Osim Bluetooth modula tu je i inkrementalni enkoder (\texttt{Incremental encoder}) pomoću kojeg je moguće mjeriti kutnu brzinu zamašnjaka. Funkcionalnost inkrementalnog enkodera u trenutku pisanja rada još nije implementirana. Nadalje, na pločici se nalazi i IMU modul (\texttt{IMU unit module}) oznake TODO koji na sebi sadrži IMU senzor. Na kraju, pločica sadrži i utor za Nucleo pločicu (\texttt{STM32 Nucleo-32 board}) koju je moguće jednostavno izvaditi i natrag staviti. Radi lakšeg razvijanja programske podrške na pločici se nalazi nekoliko statusnih LED dioda.  

            \begin{figure}[htb]
            \centering
            \includegraphics[width=1.0\textwidth]{images/plocica_shema.png}
            \caption{Shema tiskane pločice ADCS sustava.}
            \label{fig:plocica_shema}
            \end{figure}
        }

        \subsection{Izabrani senzori}{
            Zbog prirode algoritama za procjenu orijentacije (vidi poglavlje \ref{subsection:odredjivanje_orijentacije}), potrebno je imati 2 neovisna troosna senzora. U praksi bi to npr. mogli biti senzor Sunca i magnetometar. U našem slučaju, radi jednostavnije izvedbe testnog sustava i samog testiranja algoritama za estimaciju, izabrali smo akcelerometar kao prvi, odnosno magnetometar kao drugi senzor. 

            Oba troosna senzora (plus troosni žiroskop) nalaze se u jedinstvenom kućištu tzv. IMU \engl{Inertial Measurement Unit} senzora oznake \emph{MPU9250} \cite{mpu9250_datasheet}. U našem sustavu koristimo Arduino modul sa MPU9250 senzorom (slika \ref{fig:mpu9250}).

            \begin{figure}[htb]
            \centering
            \includegraphics[width=0.5\textwidth]{images/mpu9250.jpg}
            \caption{Arduino modul sa MPU9250 senzorom \cite{wolles_mpu9250}.}
            \label{fig:mpu9250}
            \end{figure}

            \subsubsection{Akcelerometar}{
                Akcelerometar mjeri trenutnu akceleraciju na tijelo/senzor \cite{}. Ako tijelo miruje, akcelerometar će pokazivati akceleraciju od 9.81 $m/s^2$ u smjeru suprotno od središta Zemlje dok referentni vektor akceleracije Zemljine gravitacije $g$ gleda u suprotnom smjeru i jednakog je iznosa. Zbog te dvije činjenice moguće je akcelerometrom zamijeniti neki senzor koji bi se klasično koristio za estimaciju orijentacije (npr. senzor Sunca). Akcelerator donosi dva problema. Prvi problem je taj što se akcelerometar ne može koristiti u sustavima koji ne akceleriraju (npr. sateliti koji kruže oko Zemlje) jer u takvim sustavima ne postoji akceleracija. Drugi problem je što je akcelerometar senzor koji posjeduje relativno puno šuma, ali zato ne posjeduje tzv. \emph{bias}. 

                Za estimacijske algoritme koje mi koristimo nije bitno koji je senzor u pitanju i na koji način radi već je za algoritam jedino potrebno trenutni podatak sa senzora i referentni podatak. 
            }

            \subsubsection{Magnetometar}{
                Magnetometar nam za podatak daje vektor Zemljinog magnetskog polja. Pomoću dostupnih alata na internetu (vidi \cite{magnetic_model}), moguće je odrediti referentni vektor Zemljinog magnetskog polja za dato mjesto na Zemlji. Magnetometar nam je povoljan senzor utoliko što ga je moguće koristiti u Zemljinoj orbiti. Ako bi znali podatak o referentnom vektoru magnetskog polja Zemlje za svaku točku orbite i trenutnu vrijednost magnetskog polja koji bi dobili sa senzora, mogli bismo podatak dati algoritmu za estimaciju orijentacije i tako dobiti podatak o orijentaciji.
            }

            \subsubsection{Žiroskop}{
                Unutar IMU senzora nalazi se i troosni žiroskop. Kao što smo ranije spomenuli u poglavlju \ref{}, za uspješniju estimaciju orijentacije koristimo i žiroskop. Žiroskop kojeg koristimo se također nalazi unutar IMU senzora.
            }
        }

        \subsection{Izabrani aktuatori}{
            Za aktuator smo izabrali zamašnjak kojeg je moguće vidjeti na slici \ref{fig:zamasnjak}. Elektromotor zamašnjaka položen je na tiskanu pločicu ADCS sustava. Rotirajuću masu zamašnjaka smo izradili pomoću 3D printera. 

            Zamašnjak smo izabrali zato što je to aktuator kojeg je lako izraditi, dijelovi su lako dostupni, za pogon je potrebna samo električna energija koja dolazi iz baterije i drajver elektromotora je jednostavne izvedbe. 
            \begin{figure}[htb]
            \centering
            \includegraphics[width=0.7\textwidth]{images/zamasnjak.png}
            \caption{Prikaz zamašnjaka sa ADCS sustavom.}
            \label{fig:zamasnjak}
            \end{figure}
        }
    }
}


\chapter{Programska podrška ADCS sustava}{
    U ovom poglavlju bit će opisana programska podrška ADCS sustava. Opisat ćemo ugradbeno računalo koje koristimo. Objasnit ćemo organizaciju projekta (programske podrške). Navest ćemo korištene biblioteke, operacijski sustav, opisati funkcionalnost programa i u konačnici ukratko opisati postupak razvijanja programske podrške.

    Napominjemo kako sljedeći opis sustava vrijedi isključivo za stanje sustava kojeg smo zatekli u trenutku pisanja ovog rada. Cijela programska podrška nalazi se na GitHub repozitoriju autora ovdje \cite{github_repo}. Programska podrška verzionirana je pomoću Git \cite{git} programa za verzioniranje. Objašnjenja i opisi programske podrške vrijede do \emph{TODO} \emph{commit}-a \cite{git_commit}. Postoji velika vjerojatnost kako će se sustav u budućnosti mijenjati i da opis u nastavku nužno neće odgovarati stanju u budućnosti. 

    \section{Ugradbeno računalo}{
        Ugradbeno računalo koje koristimo je već gotov sustav na \emph{TODO} razvojnoj pločici \cite{}. Takav gotov sustav omogućuje nam jednostavno i brzo razvijanje programske podrške. 

        Pločica se sastoji od \emph{TODO} mikroupravljača niske potrošnje koji sadrži tzv. jedinicu za računanje aritmetike pomičnog zareza \engl{Floating point unit - FPU}. FPU nam omogućava hardversko ubrzanje aritmetike pomičnog zareza. Osim mikroupravljača, pločica sadrži ST-LINK sučelje za jednostavno \emph{debugiranje} i pohranu programa. Na samoj pločici nalazi se i korisna upravljiva LED dioda.

        Kako bi imali TODO, morali smo napraviti nekoliko preinaka na pločici. TODO objasni što smo promijenili. 
    }

    \section{Organizacija projekta}{
        Programski kod podijeljen je u nekoliko direktorija i poddirektorija. Pogled iz glavnog direktorija prikazan je u tablici \ref{tbl:pogled_iz_glavnog_direktorija}.

        Nepotrebno je ulaziti u detalje svakog direktorija već ćemo u nastavku opisati samo što se nalazi unutar \texttt{src} direktorija. Pogled iz \texttt{src} direktorija prikazan je u tablici \ref{tbl:pogled_iz_src_direktorija}.

        \begin{table}[htb]
        \caption{Pogled iz glavnog direktorija projekta.}
        \label{tbl:pogled_iz_glavnog_direktorija}
        \centering
        \begin{tabular}{lll} 
        \toprule
        Naziv & Tip & Opis \\ 
        \midrule
        cmake           & Direktorij & Pomoćne CMake datoteke \\
        docs            & Direktorij & Popratnu dokumentaciju \\
        .git            & Direktorij & Git verzioniranje \\
        .github         & Direktorij & GitHub CI server \\
        scripts         & Direktorij & Pomoćne Shell skripte \\
        src             & Direktorij & Programska podrška \\
        tests           & Direktorij & Testovi programske podrške \\
        .vscode         & Direktorij & Pomoćne datoteke za VSCode editor \\
        .clang-format   & Datoteka   & Konfiguracija za clang-format \\
        CMakeLists.txt  & Datoteka   & CMake konfiguracija projekta \\
        .gitignore      & Datoteka   & Imena direktorija/datoteka koje Git ne verzionira \\
        .gitmodules     & Datoteka   & Imena i poveznice (linkovi) na eksterne Git repozitorije \\
        LICENSE         & Datoteka   & Licenca projekta \\
        Makefile        & Datoteka   & GNU Make komande za olakšano prevođenje (kompajliranje) \\
        README.md       & Datoteka   & Opis projekta \\
        \bottomrule
        \end{tabular}
        \end{table}

        \begin{table}[htb]
        \caption{Pogled iz \texttt{src} direktorija.}
        \label{tbl:pogled_iz_src_direktorija}
        \centering
        \begin{tabular}{lll} 
        \toprule
        Naziv & Tip & Opis \\ 
        \midrule
        bsp         & Direktorij & engl. \emph{Board support package} \\
        core        & Direktorij & Postavljanje osnovnih periferija mikroupravljača (clock, memorija itd.) \\
        drivers     & Direktorij & Drajveri za periferiju (I2C, UART, PWM, tajmeri itd.) \\
        libs        & Direktorij & Eksterne biblioteke (Bluetooth, MPU9250, komplementarni filtar itd.) \\
        linker      & Direktorij & Linker datoteke \\
        mcu         & Direktorij & Drajveri za periferiju (ali za naš mikroupravljač) \\
        middlewares & Direktorij & Programski kod dretvi/aplikacije \\
        rtos        & Direktorij & Datoteke RTOS-a \\
        utils       & Direktorij & tzv. \emph{Error handling}, matematičke funkcije \\
        main.c      & Direktorij & Main (glavna) funkcija \\
        main.h      & Direktorij & Main (glavna) funkcija \\
        \bottomrule
        \end{tabular}
        \end{table}
    }
    
    \section{Korištene biblioteke}{
        U sklopu programske podrške koristimo i nekoliko biblioteka:

        \begin{itemize}
        \item comp\_filt,
        \item mpu9250,
        \item optimal\_request,
        \item printf,
        \item zs040.
        \end{itemize}

        Neke biblioteke su samostalni GitHub repozitoriji i u projekt su dodani kao tzv. Git submoduli. Primjer su \texttt{mpu9250} biblioteka koja upravlja MPU9250 IMU senzorom, \texttt{optimal\_request} biblioteka koja sadrži kod za Optimal-REQUEST algoritam, \texttt{printf} biblioteka koja sadrži jednostavniju verziju klasične \texttt{printf} funkcije i \texttt{zs040} biblioteka koja upravlja Bluetooth modulom.

        \texttt{comp\_filt} biblioteka sadrži programski kod za Komplementarni filtar i jedina je biblioteka koja nema svoj zasebni GitHub repozitorij ali ta se činjenica može promijeniti u budućnosti. 

        Pogodnost eksternih GitHub repozitorija je u tome što to mogu biti samostalni repozitoriji koje netko drugi može uzeti i ukomponirati u svoj projekt neovisno o tome na kojemu se sustavu pokreće. Netko tko želi koristiti npr. Optimal-REQUEST biblioteku može jednostavno dodati GitHub repozitorij kao submodul i pokrenuti to na svom sustavu. 
    }

    \section{Operacijski sustav i funkcionalnost}{
        Zahtjev sustava je da u realnom vremenu obavlja nekoliko stvari istovremeno:

        \begin{enumerate}
        \item Estimacija orijentacije,
        \item Kontrola orijentacije,
        \item Komunikacija.
        \end{enumerate}

        \subsection{Operacijski sustav i dretve}{
            Radi jednostavnijeg razvijanja programske podrške odlučili smo se za operacijski sustav u stvarnom vremenu \engl{Real Time Operating System - RTOS}. Zbog popularnosti i široke zajednice, izabrali smo FreeRTOS operacijski sustav. FreeRTOS je besplatni operacijski sustav otvorenog koda koji ima podršku za mikroupravljač kojeg koristimo \cite{freertos}. 
    
            Sustav smo podijelili u nekoliko dretvi koje se izvršavaju paralelno:
            \begin{enumerate}
            \item Dretva budnosti \engl{Alive thread},
            \item Dretva za estimaciju i kontrolu orijentacije,
            \item Dretva za komunikaciju.
            \end{enumerate}

            \subsubsection{Dretva budnosti}{
                Dretva budnosti je jednostavna dretva u kojoj se periodički pali i gasi LED statusna dioda koja se nalazi na samoj Nucleo pločici. Dretva budnosti nam omogućuje vizualnu provjeru rada sustava. Ako iz nekog razloga izvršavanje sustava prestane, jasno se na oko može vidjeti kako statusna LED-ica ne treperi. To je znak kako je sustav otišao u neodređeno stanje. Pseudokod dretve budnosti nalazi se u nastavku:

                \begin{lstlisting}[language=C]
while(True) {
    promijeni_stanje_statusne_ledice();
    čekaj_100_milisekundi();
}               \end{lstlisting}
            }
    
            \subsubsection{Dretva za estimaciju i kontrolu orijentacije}{
                Dretva za estimaciju i kontrolu orijentacije (ADCS dretva) sadrži programski kod koji periodički estimira trenutnu orijentaciju i na temelju toga upravlja regulacijskim petljama. Pseudokod ADCS dretve nalazi se u nastavku:
        
                \begin{lstlisting}[language=C]
pričekaj_3_sekunde_za_stabilizaciju_satelita();

inicijaliziraj_imu();
inicijaliziraj_komplementarni_filtar();
inicijaliziraj_zamašnjak();
inicijaliziraj_regulacijske_petlje();

while(True) {
  imu_mjerenje = izmjeri_imu_vrijednosti();

  vrsta_regulacije = potraži_vrstu_regulacije();
  if (vrsta_regulacije == REGULACIJA_ORIJENTACIJE) {
    kvaternion = estimiraj_orijentaciju(imu_mjerenje);

    izmjereni_kut = izračunaj_eulerov_kut_z(kvaternion);
    željeni_kut = potraži_željeni_eulerov_kut();
    reguliraj_orijentaciju_oko_z_osi(željeni_kut, izmjereni_kut);

    izlaz_regulatora = potraži_izlaz_pid_regulatora_orijentacije();

    pošalji_preko_bluetootha(željeni_eulerov_kut, 
      izmjereni_eulerov_kut, 
      izlaz_regulatora);

  } else if (vrsta_regulacije == REGULACIJA_KUTNE_BRZINE) {
    željena_kutna_brzina = potraži_željenu_kutnu_brzinu();
    reguliraj_kutnu_brzinu_oko_z_osi(željena_kutna_brzina, 
      imu_mjerenje.žiroskop.z);

    izlaz_regulatora = potraži_izlaz_pid_regulatora_kutne_brzine();
    pošalji_preko_bluetootha(željena_kutna_brzina, 
      imu_mjerenje.žiroskop.z, 
      izlaz_regulatora);

  } else if (vrsta_regulacije == BEZ_REGULACIJE) {
      kvaternion = estimiraj_orijentaciju(imu_mjerenje);
      kutovi = izračunaj_eulerove_kutove(kvaternion);
      željeni_kut = potraži_željeni_eulerov_kut();
      zamašnjak = potraži_duty_cycle_zamašnjaka();

      pošalji_preko_bluetootha(kutovi, 
        imu_mjerenje.žiroskop, 
        zamašnjak);
  }

  čekaj_100_milisekundi();
}               \end{lstlisting}

                Vidimo da na samom početku ADCS dretve imamo čekanje od 3 sekunde ne bi li se satelit smirio. Razlog tomu je što na samom početku inicijalizacije sustava imamo kalibraciju žiroskopa. Kalibracija žiroskopa nastoji pronaći konstantni bias ali pod uvjetom da je satelit u stacionarnom stanju, odnosno da je kutna brzina jednaka 0 za svaku os. Tih 3 sekunde čekanja nam omogućuje da nakon ponovnog pokretanja sustava, imamo vremena satelit položiti negdje i pričekati da prođe vrijeme kalibracije.

                Nakon 3 sekunde čekanja slijedi inicijalizacija IMU jedinice, komplementarnog filtra, zamašnjaka i obje regulacijske petlje.

                Nakon inicijalizacije slijedi beskonačna petlja gdje se prvo potražuje mjerenje s IMU jedinice. IMU jedinica dat će podatke o vektorima akceleracije, magnetskog polja i kutne brzine za svaku os. 

                Jednom kada smo dobili vrijednosti mjerenja, slijedi potraživanje stanja regulacije. Postoje 3 stanja regulacije: 

                \begin{enumerate}
                    \item Regulacija orijentacije oko z-osi,
                    \item Regulacija kutne brzine oko z-osi,
                    \item Bez regulacije.
                \end{enumerate}

                Sva 3 stanja zapravo dolaze iz komunikacijske dretve koja omogućuje da korisnik u bilo koje vrijeme odabere jedan od tih stanja (vidi komande koje pruža komunikacijska dretva). Prvotno stanje regulacije (nakon svakog ponovnog pokretanja) je stanje bez regulacije.

                Stanje regulacije orijentacije započinje estimacijom orijentacije. Funkcija koja estimira orijentaciju kao podatak prima mjerenja iz IMU jedinice (koristi sva 3 podatka), a na izlaz daje procijenjeni kvaternion. Funkcija za estimaciju orijentacije u pozadini poziva komplementarni filtar koji računa kvaternion. Nakon što smo dobili kvaternion, računamo Eulerov kut oko z-osi. Nadalje, potražujemo željeni Eulerov kut koji zapravo dolazi također iz komunikacijske dretve jer korisnik u svakom trenutku može promijeniti željeni kut. Pomoću izračunatog (estimiranog) kuta i željenog kuta pozivamo funkciju za regulaciju orijentacije. U pozadini ta funkcija poziva PID regulacijsku petlju za kontrolu orijentacije (kuta). Na samom kraju dretve šalju se podaci preko Bluetooth-a korisniku o vrijednosti željenog (referentnog) Eulerovog kuta, trenutnog (estimiranog) Eulerovog kuta i izlaz PID regulatora. Te tri vrijednosti važne su nam u proučavanju odziva satelita prilikom regulacije orijentacije. 

                Stanje regulacije kutne brzine započinje pozivanjem funkcije regulacijske petlje za regulaciju kutne brzine. Funkcija koja regulira kutnu brzinu kao podatak prima mjerenja iz žiroskopa (samo z-os) i željenu kutnu brzinu. Željenu kutnu brzinu potražujemo iz komunikacijske dretve koja omogućava da korisnik u svakom trenutku promijeni željenu kutnu brzinu. Funkcija za regulaciju kutne brzine u pozadini poziva funkciju za PID regulator kutne brzine. Pri samom kraju dretve šaljemo podatke korisniku o vrijednosti željene (referentne) kutne brzine, trenutne (izmjerene) kutne brzine i izlaz PID regulatora. 

                Stanje bez regulacije započinje estimiranjem orijentacije, računanjem Eulerovih kutova i čitanjem vrijednosti Duty Cycle-a PWM-a zamašnjaka (trenutne i referente). Sve to na kraju zajedno pošaljemo korisniku. Ti podaci su nam važni za snimanje odziva satelita na tzv. Step pobudu (vidi poglavlje \ref{TODO}) i računanje koeficijenata pojednostavljenog modela (vidi poglavlje \ref{TODO}).

                Na samom kraju dretve slijedi čekanje od 100 milisekundi. Obratimo pažnju i na to da regulacijske petlje ovise o periodu izvođenja ADCS dretve.
            }

            \subsubsection{Dretva za komunikaciju}{
                Dretva za komunikaciju sadrži programski kod koji je zaslužan za komunikaciju sa satelitom. Komunikacija sa satelitom odvija se u oba smjera: od korisnika prema sustavu i od sustava prema korisniku. Kao što je rečeno u poglavlju \ref{}, komunikacija se vrši preko Bluetooth-a i to nam omogućuje da jednostavno možemo pristupiti satelitu pomoću računala koji sadrži Bluetooth sučelje ili preko pametnog telefona. Za komunikaciju preko računala koristimo program TODO koji preko COM porta računala, jednostavno komunicira sa satelitom. Za komandno sučelje na pametnom telefonu preporučujemo aplikaciju TODO. 

                Takva dretva nam omogućava slanje podataka o trenutnom stanju satelita npr. trenutne orijentacije, kutne brzine, izlaza upravljačkih regulatora i sl. Druga bitna funkcionalnost dretve je mogućnost upravljanja satelitom sa strane korisnika. Popis svih podržanih komandi upravljanja satelitom prikazani su u tablici \ref{tbl:lista_komandi}.

                \begin{table}[htb]
                \caption{Lista komandi, njihov opis i primjer korištenja. \textbf{Obratite pažnju kako svaka komanda u primjeru mora završavati sa '\textbackslash n' ASCII znakom.}}
                \label{tbl:lista_komandi}
                \centering
                \begin{tabular}{lll} 
                \toprule
                Komanda & Opis & Primjer korištenja \\ 
                \midrule
                echo                     & Vraća nazad korisniku sve primljeno          & \texttt{echo(ADCS)}                           \\
                help                     & Ispisuje sve podržane komande                & \texttt{help()}                               \\
                reset                    & Ponovno pokretanje sustava                   & \texttt{reset()}                              \\
                set\_sending             & Omogućuje/onemogućuje slanje poruka          & \texttt{set\_sending(1)}, (ili \texttt{0})    \\
                set\_reg\_angvel         & Pokreće regulaciju kutne brzine              & \texttt{set\_reg\_angvel()}                   \\
                set\_reg\_attitude       & Pokreće regulaciju orijentacije              & \texttt{set\_reg\_attitude()}                 \\
                set\_reg\_no\_reg        & Onemogućuje regulaciju                       & \texttt{set\_reg\_no\_reg()}                  \\
                set\_pid\_p              & Postavlja P član aktivnog PID regulatora     & \texttt{set\_pid\_p(3.14)}                    \\
                set\_pid\_i              & Postavlja I član aktivnog PID regulatora     & \texttt{set\_pid\_i(15)}                      \\
                set\_pid\_d              & Postavlja D član aktivnog PID regulatora     & \texttt{set\_pid\_d(92.2)}                    \\
                set\_pid\_v              & Postavlja V član aktivnog PID regulatora     & \texttt{set\_pid\_v(6.67)}                   \\
                set\_ref\_angle\_z       & Postavlja ref. kut z-osi (stupnjevi)         & \texttt{set\_ref\_angle\_z(30)}               \\
                set\_ref\_angvel\_z      & Postavlja ref. kutnu brzinu z-osi (rad/s)    & \texttt{set\_ref\_angvel\_z(-1.2)}            \\
                set\_rw\_duty\_cycle\_z  & Postavlja Duty cycle zamašnjaka              & \texttt{set\_rw\_duty\_cycle\_z(99)}          \\
                \bottomrule
                \end{tabular}
                \end{table}

                Obratimo pažnju kako svaka komanda da bi je satelit primio, mora završavati s ASCII znakom \texttt{\textbackslash n} odnosno novim redom \engl{new line, line feed - LF}. Također, komunikacijska dretva će ignorirati bilo koju drugu komandu koja nije opisana u tablici \ref{tbl:lista_komandi}.

                Svi parametri (npr. kutna brzina, Eulerovi kutovi i sl.) koje želimo slati preko Bluetooth sučelja prolaze kroz komunikacijsku dretvu. Kako ADCS dretva jedina ima informaciju o npr. trenutnoj orijentaciji, morali smo implementirati način komunikacije između dretvi. Način komunikacije između dretvi odvija se pomoću TODO. Prikaz svih dretvi s međusobnim komunikacijskim kanalima nalazi se na slici TODO. 
                
                Komunikacijska dretva u normalnom stanju čeka poruke za slanje i tako ne "krade" procesorsko vrijeme ostalim dretvama. Jednom kada je dretva primila poruku za slanje, dretva preuzima procesorsko vrijeme i šalje preko UART sučelja dane joj podatke. Jednom kad su svi podaci poslani, komunikacijska dretva ponovno ulazi u stanje čekanja.
                
                Na isti način se odvija komunikacija u suprotnom smjeru od korisnika prema satelitu. Jednom kad Bluetooth sučelje na satelitu primi podatak od korisnika, komunikacijska dretva se budi iz stanja pripravnosti i pohranjuje sve podatke. Jednom kada je prikupila sve podatke, komunikacijska dretva će pročitati primljenu komandu i na temelju toga izvršiti funkcije (vidi komande).

                Pseudokod komunikacijske dretve nalazi se u nastavku:

                \begin{lstlisting}[language=C]
inicijaliziraj_bluetooth();

while (True) {
    if (provjeri_ima_li_poruka_za_slanje()) {
        poruka = potraži_poruke_od_drugih_dretvi();
        pošalji_poruku_preko_bluetootha(poruka);
    }

    if (provjeri_ima_li_poruka_od_korisnika()) {
        poruka = potraži_primljenu_poruku();

        komanda, arg = potraži_komandu_i_argument_iz_poruke(poruka);

        for (podržana_komanda : podržane_komande) {
            if (podržana_komanda == komanda) {
                funkcija = potraži_funkciju(komanda);
                funkcija(arg);
            }
        }
    }

    raspusti_dretvu();
}           \end{lstlisting}

                Funkcije koje provjeravaju ima li poruka za slanje ili poruke koje smo primili od korisnika nisu blokirajuće. Na kraju beskonačne petlje nalazi se funkcija koja raspušta izvršavanje dretve i daje procesorsko vrijeme drugim dretvama. Izvršavanje komunikacijske dretve odvija se onda kada sve dretve budu u stanju čekanja. Primanje komandi odvija se pomoću UART prekidne rutine gdje na svaki primljeni znak s Bluetooth modula, mikroupravljač odlazi u prekidnu rutinu i puni spremnik znakova koji sadrži sve primljene znakove. Jednom kada komunikacijska dretva dobije procesorsko vrijeme ona će provjeriti ima li primljenih znakova u spremniku znakova i na temelju toga će ići u daljnju obradu tih znakova. 

                % todo: dodaj sliku svih dretvi i mogucnost medjusobne komunikacije
            }
        }
    }

    \section{Razvoj}{
        Ukratko ćemo objasniti na koji način se razvija programska podrška. Programska podrška napisana je u C programskom jeziku standarda C99 \cite{c99}. Sva programska logika nalazi se unutar već spomenutog \texttt{src} direktorija, a testovi se razvijaju unutar \texttt{tests} direktorija. 
        
        Razvoj počinje tzv. kloniranjem projekta koji se nalazi na GitHub repozitoriju \cite{github_repo} pomoću Git programa. Kloniranje započinje komandom \texttt{git clone https://github.com/IvanVnucec/cubesat-adcs --recursive}. Git će pohraniti sve datoteke koje se nalaze na repozitoriju lokalno na disk razvojne mašine. Napominjemo kako nije nužno klonirati projekt pomoću Git programa već se program može jednostavno preuzeti s GitHub repozitorija i ručno spremiti na disk. Autor preporučuje kloniranje Git-om jer je to jednostavniji postupak.  

        Za pomoć pisanju programskog koda autor programske podrške odlučio se za \emph{VSCode} tekstualni uređivač \cite{vscode}. VSCode pruža mogućnost jednostavnijeg potraživanja programskog koda, bojanje programskih linija i slično.
        
        Jednom kada smo implementirali željenu funkcionalnost, slijedi postupak prevođenja \engl{compilation}. Za prevođenje koristimo \emph{GNU Arm Embedded Toolchain} prevodilac  verzije \emph{v10.3-2021.10} \cite{gnu_arm_toolchain}. Kao graditelj sustava \engl{build system} koristimo \emph{CMake} verzije \emph{v3.18.4} \cite{cmake}. Pomoću CMake-a je moguće jednostavnije postaviti sustav za prevođenje neovisno o tome koju vrstu operacijskog sustava koristimo na razvojnoj mašini (npr. Windows, Ubuntu itd.). Prilikom prvog prevođenja potrebno je postaviti projekt pomoću GNU Make programa tako što pokrenemo komandu \texttt{make setup\_cmake}. Nakon postavljanja projekta moguće je pokrenuti prevođenje pomoću komande \texttt{make build}, ili kraće \texttt{make}. 

        Za provjeru rada programske podrške (debugiranje), koristimo \emph{OpenOCD} \cite{openocd} program koji u pozadini otvori GDB \cite{gdb} server na kojeg se je moguće spojiti. Debugiranje započinjemo unutar VSCode-a tako što otvorimo \emph{Run and Debug} karticu i pokrenemo debugiranje pritiskom na zelenu strelicu. VSCode će pokrenuti prevođenje i pohranu programa na mikroupravljač pod uvjetom da je Nucleo pločica priključena na razvojno računalo pomoću USB kabela. Postavimo li npr. glavnu funkciju zaustavnu točku \engl{breakpoint}, program bi morao na trenutak prestati s izvođenjem i korisniku dati pristup vrijednostima svih varijabli koje program koristi. Pritiskom na tipku nastavi \engl{Continue}, program bi trebao nastaviti s izvođenjem. Ako postupkom debugiranja vidimo neželjeno ponašanje programa, debugiranje se može prekinuti i greška se može prepraviti. Jednom kada smo prepravili grešku, možemo opet na isti način pokrenuti postupak debugiranja.

        Jednom kada smo zadovoljili sve zahtjeve programske podrške, preporučuje se da se programski kod formatira. Formatiranje programskog koda je važno ako na istom sustavu surađuje više osoba. Svaka osoba ima svoj način pisanja programske podrške i zbog toga može doći do otežane čitljivosti koda. Da bi tome pribjegli koristimo \emph{ClangFormat} program za formatiranje programskog koda \cite{clang_format}. Formatiranje programskog koda pokreće se komandom \texttt{make clang\_format}.

        Jednom kada smo napravili sve postupke kod je moguće verzionirati pomoću Git-a postupkom koji je objašnjen ovdje \cite{git_commit_instrukcije} ili poslati direktno autoru projekta.

        Kao rubriku zanimljivosti navest ćemo broj linija programskog koda za svaki direktorij unutar \texttt{src} direktorija. Broj linija koda nalazi se u tablici \ref{tbl:broj_linija}.

        \begin{table}[htb]
        \caption{Broj linija koda za svaki direktorij unutar \texttt{src} direktorija.}
        \label{tbl:broj_linija}
        \centering
        \begin{tabular}{lll} 
        \toprule
        Direktorij & C99 & Assembly \\ 
        \midrule
        drivers     &     63330    &   0    \\
        libs        &     6776     &   0    \\
        rtos        &     13229    &   0    \\
        middlewares &     1444     &   0    \\
        core        &     633      &   283  \\
        mcu         &     367      &   0    \\
        bsp         &     125      &   0    \\
        utils       &     53       &   0    \\
        top\_dir    &     30       &   0    \\
        \bottomrule
        \end{tabular}
        \end{table}
    }
}

\chapter{Eksperimentalna verifikacija ADCS sustava}{
    U ovom poglavlju objasnit ćemo testni sustav za eksperimentalnu verifikaciju programske podrške. 

    \section{Opis sustava i korištenih alata}{
        Verifikacijski sustav sastoji se od nekoliko dijelova. Prvi i najvažniji je tzv. zračni ležaj sa pripadajućom plastičnom kuglom prikazan na slici \ref{fig:fotka_zracni}.

        \begin{figure}[htb]
        \centering
        \includegraphics[width=0.6\textwidth]{images/fotka_zracni.jpg}
        \caption{Prikaz zračnog ležaja sa pripadajućom kuglom.}
        \label{fig:fotka_zracni}
        \end{figure}

        Zračni ležaj sastoji se od 3D isprintane polusferične podloge koja je napajana stlačenim zrakom koji dolazi iz kompresora zraka. Stlačeni zrak koji dolazi na površinu zračnog ležaja stvara tanki film zračne struje koji smanjuje trenje između ležaja i plastične kugle. Plastična kugla sastoji se od vanjske ljuske i plastičnog držača satelita. Plastična vanjska ljuska sastavljena je od dvije polusfere koje je vrlo lako moguće sastaviti i rastaviti. Plastični držač izrađen je od laserski rezanog plexiglasa i prikazan je na slici \ref{fig:drzac}. Posebnim navojnim odstojnicima satelit je pričvršćen na plastični držač \cite{zracni_lezaj}.

        \begin{figure}[htb]
        \centering
        \includegraphics[width=0.6\textwidth]{images/drzac.png}
        \caption{Prikaz plastičnog držača satelita \cite{zracni_lezaj}.}
        \label{fig:drzac}
        \end{figure}

        Za komuniciranje sa satelitom koristimo računalo ili pametni mobilni telefon koji sadrže Bluetooth sučelje. Za klasično komandno sučelje na računalu koristimo program \emph{Termite} \cite{termite} prikazan na slici \ref{fig:termite}, a na mobilnom telefonu aplikaciju \emph{Serial Bluetooth Terminal} \cite{mobilna_app} prikazanu na slici \ref{fig:mobilna_app}.

        \begin{figure}[htb]
        \centering
        \includegraphics[width=0.8\textwidth]{images/termite.png}
        \caption{Prikaz \emph{Termite} aplikacije za pristup COM komandnom sučelju \cite{termite}.}
        \label{fig:termite}
        \end{figure}

        Podešavanje \emph{Termite} programa je vrlo jednostavno. Postupak započinje spajanjem računala i satelita preko Bluetooth sučelja čiji postupak nećemo objašnjavati jer ovisi o tipu i verziji operacijskog sustava. Nakon što smo se priključili sa računalom na Bluetooth satelita, odabiremo pripadajući COM port unutar \emph{Termite} aplikacije. Ako je sve uspješno obavljeno moguće je vidjeti podatke koje dolaze sa satelita. Jednako tako moguće je slanje komandi prema satelitu. Važno je napomenuti kako svaka komanda mora završavati sa $'\backslash n'$ znakom što je moguće podesiti kao postavku unutar \emph{Termite} programa. 
        
        Podešavanje \emph{Serial Bluetooth Terminal} mobilne aplikacije još je jednostavnije. Spajanje na satelit je vrlo jednostavno i neće biti objašnjeno. Nakon uspješnog spajanja unutar aplikacije pojavljuju se podaci sa satelita. Slanje komandi prema satelitu jednostavno se šalju upisivanjem u komandi u za to predviđeni prozor. Napominjemo kako je u postavkama aplikacije potrebno podesiti da svaka komanda završava sa $'\backslash n'$ znakom. 

        \begin{figure}[htb]
        \centering
        \includegraphics[width=0.4\textwidth]{images/mobilna_app.png}
        \caption{Prikaz \emph{Serial Bluetooth Terminal} Android aplikacije \cite{mobilna_app}.}
        \label{fig:mobilna_app}
        \end{figure}
        
        Ako želimo grafički prikazivati vrijednosti koje dolaze sa satelita, to možemo učiniti sa \emph{SerialPlot} programom \cite{serialplot}. \emph{SerialPlot} program pruža različite mogućnosti: primanje i slanje komandi, grafičko prikazivanje vrijednosti, spremanje podataka u CSV datoteku i još mnogo toga. Program je besplatan i otvorenog koda. Sučelje \emph{SerialPlot} programa prikazano je na slici \ref{fig:serialplot}.

        \begin{figure}[htb]
        \centering
        \includegraphics[width=0.8\textwidth]{images/serialplot.png}
        \caption{Prikaz \emph{SerialPlot} aplikacije za grafički prikaz vrijednosti sa COM sučelja \cite{serialplot}.}
        \label{fig:serialplot}
        \end{figure}

        Postupak spajanja \emph{SerialPlot} programa i satelita praktično je isto kao i spajanje sa Termite programom. Prilikom uspješnog spajanja na ekranu se pojavljuju grafičke vrijednosti sa satelita. U programu je moguće odabrati automatsku detekciju broja podataka u ovisnosti o tome na koji način šaljemo vrijednosti. U našem slučaju vrijednosti šaljemo kao cijele brojeve odvojene znakom zareza. Takvu postavku moramo odabrati i unutar \emph{SerialPlot} programa. Nakon uspješnog podešavanja postavki na ekranu bi se trebali prikazivati grafovi vrijednosti koje dolaze sa satelita. Svaku pojedinačnu vrijednost moguće je sakriti ili prikazati. 
    }

    \section{Određivanje parametara}{
        \subsection{Kinematički model}{
            Kao što smo već pisali u poglavlju \ref{section:osnovni_parametri_satelita}, pojednostavljeni kinematički model satelita (jednadžba \ref{eq:laplace_prvi_red}) definiramo kao:

            \begin{equation}
                H(s) = \frac{Y(s)}{X(s)} = K \frac{1/\tau}{s + 1/\tau},
            \end{equation}

            gdje su parametri $K$ i $\tau$ nepoznanice koje je potrebno odrediti. 
            
            Postupak određivanja parametara započinje pokretanjem ADCS sustava. Nakon što se ADCS sustav pokrenuo i kada je prošla sva inicijalizacija, satelit postavljamo na zračni ležaj i to tako da satelit miruje. Nakon što smo uspostavili sve uvjete za testiranje, možemo krenuti na snimanje odziva sustava na tzv. step pobudu.

            Snimanje sustava na step pobudu započinje pokretanjem programa TODO pomoću kojeg je vrlo lako moguće snimiti sve vrijednosti u CSV format datoteke koje je potom lako analizirati pomoću alata kao što je MATLAB. Prilikom pokretanja TODO programa odaberemo COM port na kojemu je moguće dobiti podatke iz satelita. Namjestimo li automatsko primanje svih parametara na ekranu ćemo vidjeti 3 vrste signala: estimirani Eulerovi kutovi, mjerenje sa žiroskopa i Duty Cycle zamašnjaka. Pomoću mjerenja žiroskopa i vrijednosti DC zamašnjaka moguće je odrediti kinematički model satelita. Primijetimo kako u jednadžbi \ref{TODO} kao ulazni parametar sustava imamo DC zamašnjaka, a kao izlaz kutnu brzinu satelita. 

            Odziv sustava na dvije pobude prikazan je na slici \ref{fig:step_response}. Oko 7. sekunde poslali smo satelitu naredbu da zavrti zamašnjak na 50\% DC nakon koje kutna brzina naglo raste do nekih 3 rad/s. Primijetimo kako oko 20. sekunde kutna brzina počinje blago padati. Razlog tomu su blage nepravilnosti zračnog ležaja koje uzrokuju parazitni moment koji onda usporava satelit. Modeliranje parazitnog momenta nismo uzimali u obzir radi jednostavnosti jednadžbi. Postupak snimanja step pobude napravili smo još jednom od 60. do 85. sekunde.

            \begin{figure}[htb]
            \centering
            \includegraphics[width=1.0\textwidth]{other/step_response.png}
            \caption{Odzivi sustava na Step pobudu sa DC zamašnjaka na 50\%.}
            \label{fig:step_response}
            \end{figure}

            Eksperimentalno se dade pokazati kako odziv na step pobudu satelita u Laplaceovoj domeni možemo zapisati kao:

            \begin{equation}
                Y(t) = H(s)X(s) = K \frac{1/\tau}{s + 1/\tau} \; \frac{1}{s},
            \end{equation}

            odnosno u vremenskoj domeni:

            \begin{equation}
                y(t) = (h \ast x)(t) = K (1 - e^{-\frac{t}{\tau}}),
            \end{equation}

            Parametar $K$ označuje koeficijent pretvorbe između DC zamašnjaka i izlazne kutne brzine. Parametar $\tau$ označava koeficijent tromosti sustava: što je sustav tromiji to je parametar veći, i obrnuto.

            Pomoću MATLAB alata i pripadajućih MATLAB skripti koje se nalaze unutar \texttt{docs/step\_response} direktorija \cite{link_na_fit}, moguće je na jednostavan način dobiti $K$ i $\tau$ parametre kinematičkog modela. Primjer dobivanja parametara preko MATLAB skripti moguće je vidjeti na slici \ref{fig:fittana_funkcija}. Napisana skripta prvo učitava već snimljene podatke (vidi sliku \ref{fig:step_response}) i izlučuje trenutak eksponencijalnog rasta kutne brzine. Na temelju tog vremenskog odsječka skripta će kao rezultat dati $K$ i $\tau$ koeficijente. 

            \begin{figure}[htb]
            \centering
            \includegraphics[width=1.0\textwidth]{other/fittana_funkcija.png}
            \caption{Graf odziva na Step i graf funkcije kojoj smo pronašli nepoznate parametre.}
            \label{fig:fittana_funkcija}
            \end{figure}

            Spomenute MATLAB skripte osim estimacije parametara kao rezultat daju i kinematičke modele. Prvi kinematički model odnosi se na modeliranje kutne brzine, a drugi na modeliranje orijentacije (kuta). Kinematički model za modeliranje orijentacije dobivamo integriranjem modela kutne brzine satelita (jednadžba \ref{eq:laplace_prvi_red}) i definiramo ga kao:

            \begin{equation}
                Y(t) = \frac{1}{s} \; H(s) =  \frac{1}{s} \; K \frac{1/\tau}{s + 1/\tau}.
            \end{equation}
        }
    
        \subsection{PID regulator}{
            \subsubsection{Regulator kutne brzine}{
                Pomoću spomenuta dva modela možemo eksperimentalno odrediti koeficijente PID regulatora. Otvorimo TODO MATLAB alat u kojemu prvo unosimo prvi kinematički model za modeliranje kutne brzine. Podešavanjem kliznika moguće je PID regulator ugoditi na željeni odziv u ovisnosti o tome želimo li brži odziv s relativno velikim izdizanjem \engl{overshoot} ili sporiji odziv s blažim izdizanjem. U drugom prozoru moguće je otvoriti graf na kojemu je prikazan izlaz regulatora \engl{controller effort}. Pomoću takvog grafa možemo provjeriti da nam izlaz iz regulatora ne prelazi DC zamašnjaka jer se u protivnom izlaz regulatora ograničava na 100\% DC zamašnjaka što uvodi dodatne nelinearnosti u sustav. 
            }

            \subsubsection{Regulator orijentacije}{
                Jednaki postupak određivanja koeficijenata PID regulatora vrijedi i za regulaciju orijentacije (kuta). Skripta će na izlazu, osim modela kutne brzine, dati i model orijentacije. Taj model možemo također učitati u TODO MATLAB-ov alat i po potrebi ugoditi koeficijente PID regulatora.
            }
        }
    }

    \section{Rezultati verifikacije}{
        Rezultate parametrizacije PID regulatora snimili smo prilikom testiranja sustava. Na slici \ref{fig:ang_vel_reg_user} prikazan je odziv sustava na promjenu željene kutne brzine u nekoliko navrata. U prvom navratu korisnik je u 5. sekundi komandom zadao željenu kutnu brzinu od 1 rad/s. Kutna brzina naglo počinje rasti sve dok satelit ne dosegne željenu brzinu oko 20. sekunde. Neposredno nakon toga satelitu je poslana naredba za referentnu brzinu od 0 rad/s i to se jasno može vidjeti naglim ali kontroliranim padom do 0 rad/s. Jednaki postupak izveli smo u 55. sekundi gdje smo satelitu zadali kutnu brzinu od 1 rad/s ali sada u suprotnom smjeru. Žuto obojana linija predstavlja vrijednosti izlaza regulatora. Izlaz regulatora direktno utječe na DC zamašnjaka. Negativna vrijednost označava suprotni smjer vrtne zamašnjaka. Primijetimo kako graf izlaza regulatora ne prelazi $\pm$100\%. Primijetimo još činjenicu da oko 30. i 85. sekunde postoji blago nadvišenje s obzirom na referentnu kutnu brzinu koje traje relativno dugo. Uzrok tog nadvišenja je mala greška koja dolazi na ulaz PID regulatora. PID regulator množi tu malu grešku sa K članom ali to i dalje nije dovoljno za značajniju korekciju. O D članu nema govora jer je sustav u stacionarnom stanju i ne postoje visokofrekventne komponente na koje bi D član reagirao. Primijetimo kako je izlaz PID regulatora u početku relativno mali i da raste s vremenom. To je upravo I član koji integrira malu grešku i koji će nakon nekoliko vremena postati značajan faktor izlaza regulatora koji onda vrši korekciju kutne brzine. 

        \begin{figure}[htb]
        \centering
        \includegraphics[width=1.0\textwidth]{other/ang_vel_reg_user.png}
        \caption{Odzivi sustava na promjenu referentne kutne brzine.}
        \label{fig:ang_vel_reg_user}
        \end{figure}

        Na drugoj slici \ref{fig:ang_vel_reg_dist} vidimo odziv regulatora na vanjske smetnje gdje smo u nekoliko navrata ručno pomaknuli satelit i tako mu doveli smetnju. Dva su trenutka smetnje: jedna oko 10. sekunde i druga oko 32. sekunde. Prva smetnja ima različitu orijentaciju u odnosu na drugu ne bi li ustanovili simetričnost regulacije. Vidimo kako je referentna kutna brzina cijelo vrijeme postavljena u 0 rad/s. Primijetimo kako regulator vrlo brzo vrši korekciju i da je tranzijentno vrijeme vrlo kratko. Zanimljivo je još jednom primijetiti manifestacije nepravilnosti zračnog ležaja. Jasno je vidljivo na samom početku grafa da izlaz regulatora nije 0\% već neka vrijednost. Razlog tomu je što zračni ležaj daje konstantni moment kojeg regulator pokušava kompenzirati. Primijetimo i to kako oko 40. sekunde postoji značajno nadvišenje kutne brzine što je u svakom slučaju negativna karakteristika. Daljnja eksperimentalna verifikacija mogla bi pokazati kako postoje bolji parametri PID regulatora koji bi dali bolju regulacijsku karakteristiku.

        \begin{figure}[htb]
        \centering
        \includegraphics[width=1.0\textwidth]{other/ang_vel_reg_dist.png}
        \caption{Odzivi sustava na vanjsku smetnju kutne brzine.}
        \label{fig:ang_vel_reg_dist}
        \end{figure}

        Nadalje, rezultate ugađanja PID regulatora za regulaciju kuta možemo vidjeti na slici \ref{fig:angle_reg_user} gdje je korisnik prvi put u 3. sekundi zadao referentni kut od 90 stupnjeva. Vidimo kako od tog trenutka imamo nagli skok iz mjerenog (zapravo estimiranog) kuta do oko 90 stupnjeva u 16. sekundi od početka snimanja. Nakon toga korisnik daje naredbu za referentnom orijentacijom od -90 stupnjeva. Primijetimo kako orijentacija počinje naglo padati i prelazi u drugu stranu. Na grafu je vidljivo kako postoji relativno veliko nadvišenje. Uzrok tog nadvišenja je u početku relativno velika izlazna komponenta PID regulatora gdje PID regulator zamašnjaku daje naredbu brzine vrtnje od 1500 DC. To je naravno nemoguće jer DC zamašnjaka maksimalno može biti 100\%. Razlog toliko velikog izlaza leži u relativno velikom D članu koji zapravo i mora biti velik ne bi li mogao kompenzirati veliku integralnu komponentu unutar pojednostavljenog kinematičkog modela za orijentaciju satelita (jednadžba \ref{TODO}). Bez obzira što regulator traži od zamašnjaka da se zavrti nerealnim brzinama, sustav je tako razvijen da će, slobodnim rječnikom, odrezati \engl{clamp} vrijednosti više/niže od $\pm$100\% DC zamašnjaka. Vidimo na grafu kako regulatoru treba relativno dugo vrijeme da kompenzira željeni kut. Razlog tomu je što su K i P komponente relativno malog iznosa te mora proći duže vrijeme sve dok I komponenta PID regulatora ne postane dovoljno velika da kompenzira nadvišenje. Tijekom eksperimentiranja pokušali smo različite kombinacije vrijednosti PID parametara ali s niti jednom nismo dobili zadovoljavajući rezultat. Pretpostavljamo kako se sustavi 2. reda, kao što je sustav modela orijentacije satelita, teško reguliraju konvencionalnim PID regulatorom kojega mi koristimo već se rabe neke druge vrste regulatora koji se bolje ponašanju od PID regulatora za sustave 2. reda \cite{reg_2_red}. 
        
        \begin{figure}[htb]
        \centering
        \includegraphics[width=1.0\textwidth]{other/angle_reg_user.png}
        \caption{Odzivi sustava na promjenu referentnog Eulerovog kuta.}
        \label{fig:angle_reg_user}
        \end{figure}

        Na slici \ref{fig:angle_reg_dist} možemo vidjeti odziv regulatora na vanjske smetnje gdje smo ručno sustav pomaknuli iz referentnog položaja. Prvi put u 5. sekundi pomaknuli smo sustav za oko -20 stupnjeva i vidimo po žutom grafu kako regulator reagira. Zbog visokofrekventne komponente smetnje, D član u početku prevladava sve dok I član s vremenom ne prevlada. Drugu vanjsku smetnju generirali smo u 42. sekundi gdje smo satelit ručno pomakli za oko 57 stupnjeva. Regulator kompenzira razliku referentnog i estimiranog kuta za oko 50 sekundi. I ovdje se jasno vidi negativni utjecaj nepravilnosti zračnog ležaja. 

        \begin{figure}[htb]
        \centering
        \includegraphics[width=1.0\textwidth]{other/angle_reg_dist.png}
        \caption{Odzivi sustava na vanjsku smetnju Eulerovog kuta.}
        \label{fig:angle_reg_dist}
        \end{figure}
    }
}

% TODO: dodaj zakljucak
\chapter{Zaključak}{
    Ovdje ide zaključak.
}

\bibliographystyle{templates/template}
\bibliography{literatura}

% TODO: dodaj sazetak
\begin{sazetak}{
    Ovdje ide sažetak na hrvatskom jeziku.
}

% TODO: dodaj kljucne rijeci
\kljucnerijeci{Ovdje, idu, ključne, riječi, odvojene, zarezima.}
\end{sazetak}

% TODO: Dodaj naslov na engleskom
\engtitle{Put here title on english}
% TODO: Navedite Abstract
\begin{abstract}{
    Add abstract here.
}

% TODO: Navedite Keywords
\keywords{Add, keywords, here.}
\end{abstract}

\end{document}
