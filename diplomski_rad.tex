\documentclass[times, utf8, diplomski, numeric]{templates/fer}
\usepackage{booktabs}

\begin{document}

% TODO: Navedite broj rada.
\thesisnumber{000}

\title{Programska podrška sustava za određivanje i upravljanje orijentacijom satelita}

\author{Ivan Vnučec}

\maketitle

% Ispis stranice s napomenom o umetanju izvornika rada. Uklonite naredbu \izvornik ako želite izbaciti tu stranicu.
\izvornik

% TODO: dodaj zahvalu
\zahvala{Ovdje ide zahvala}

\tableofcontents

\chapter{Uvod}
Nanosateliti CubeSat formata postaju sve popularniji u komercijalnim i znanstvenim misijama. Broj lansiranih nanosatelita raste iz godine u godinu. Sustav za određivanje i upravljanje orijentacijom satelita jedan je od najvažnijih sustava odgovoran za stabilno i ispravno usmjeravanje satelita i njegovog korisnog tereta (antene, kamere, senzori i sl.). Zadatak ovog sustava je pomoću dostupnih senzora odrediti orijentaciju satelita te je automatski ispraviti s obzirom na željenu orijentaciju koristeći se dostupnim aktuatorima. Tema diplomskog rada je razvoj programske podrške za model satelita koja omogućuje pouzdano određivanje i upravljanje orijentacijom satelita i njegovom kutnom brzinom. U radu je potrebno opisati korištene senzore i aktuatore, način matematičkog zapisa orijentacije satelita te odabranu metodu upravljanja orijentacijom i kutnom brzinom satelita. Potrebno je implementirati programsku podršku za određivanje i upravljanje orijentacijom na ugradbenom računalnom sustavu te objasniti osnovne blokove programskog koda uz pripadne grafove i dijagrame toka. U radu je također potrebno razviti osnovni kinematički model satelita, opisati optimizaciju PID regulatora za kontrolu orijentacije te izložiti rezultate eksperimentalne verifikacije implementiranih algoritama. 

\chapter{Kako citirati}
Ovako se citira na pravilan nacin \cite{oetiket2007lshort}. Ovako ide drugi citat \cite{downes2002shortams}. Evo i treceg \cite{ungar2002uvod}.

% TODO: dodaj zakljucak
\chapter{Zaključak}
Zaključak.

\bibliographystyle{templates/fer}
\bibliography{literatura}

% TODO: dodaj sazetak
\begin{sazetak}
Sažetak na hrvatskom jeziku.

% TODO: dodaj kljucne rijeci
\kljucnerijeci{Ključne riječi, odvojene zarezima.}
\end{sazetak}

\engtitle{Software for satellite attitude determination and control system}
% TODO: Navedite Abstract
\begin{abstract}
Abstract.

% TODO: Navedite Keywords
\keywords{Keywords.}
\end{abstract}

\end{document}
